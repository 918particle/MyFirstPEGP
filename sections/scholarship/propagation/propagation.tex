\documentclass[../../main.tex]{subfiles}

\begin{document}

Given that my sub-field is one collaboration, we publish papers as a collaboration.  In my first publication as a Whittier College professor, we reported on the first-ever observations of horizontally propagating RF pulses in polar ice.  Normally, RF pulses travel along curved paths in ice because the speed of light is correlated with density, which changes with depth.  Because the RF pulses are inherently wave-like, \textit{classical} electromagnetic theory states that the RF pulses should swerve downward, because the top of the wave in faster ice outpaces the bottom in the slower ice.  This ``shadowing effect'' is similar to the physics of a mirage, and removes some ice volume from which neutrinos can geometrically originate while still being detectable.  \textbf{In stark contradiction with \textit{classical} propagation theory, we have observed RF pulses traveling in horizontal paths with no apparent curvature, in multiple venues around Antarctica \cite{horizPaper}}.  \\ \hspace{0.1cm}

On a publication with 19 authors, with data collected from multiple Antarctic sites over a decade, I wrote the first third of this publication.  I collected the entire set of index of refraction data, which yields a different speed of light at a given ice depth.  I showed that if we account theoretically for density perturbations in the ice, classical horizontal RF propagation is still possible under certain conditions.  I have also shown that internal reflection layers (independently observed via radar and field samples) could also lead to horizontally propagation.  \textit{If true, the existence of horizontal RF propagation could increase the probability of UHE neutrino detection} with ARA and ARIANNA, for the simple reason that a larger volume of ice per detector module becomes a potential UHE neutrino target.  I already have one physics major, Cassady Smith, working with me on simulations that quantify these effects.  With me as advisor, Cassady received a Keck Summer Research Fellowship for Summer 2018.  I was a secondary author on two other peer-reviewed works this year on the topic of RF propagation in ice.  One paper detailed observations of a solar flare with ARA \cite{flare}, and the other quantified RF propagation measurements at the South Pole with ARA \cite{dielectric}.

\end{document}