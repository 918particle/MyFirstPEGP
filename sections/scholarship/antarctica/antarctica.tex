\documentclass[../../main.tex]{subfiles}

\begin{document}

Recently, I have facilitated merger discussions between the ARA and ARIANNA collaborations.  Program officers from the National Science Foundation (NSF) have indicated that if the ARA and ARIANNA were to merge, they would strongly support a brand new neutrino detector in Antarctica.  The NSF solicitation is entitled ``Windows on the Universe: The Era of Mutli-Messenger Astrophysics''\footnote{See the website \url{https://www.nsf.gov/funding/pgm_summ.jsp?pims_id=505593}.}, and is due December 4th, 2018.  The NSF liason is Vladimir Papitashvili (vpapita@nsf.gov) from the NSF Office of Polar Programs (OPP).  Five years after receiving my PhD, I am proud to state that we are on the verge of major detector construction, due to workshops I helped to organize at both UC Irvine and Ohio State this year.  The timing of this merger is especially impactful for Whittier College undergraduates, for soon the joint ARA and ARIANNA\footnote{Given that physics professors are not known for their ability to compromise, we are still in the process of choosing a new name.} collaboration will require a small army of undergraduate researchers. Whittier is ideally located near UC Irvine, a founding institution and the center for RF hardware and firmware development. \\ \hspace{0.1cm}

The funding scale for our Antarctic neutrino detector is approximately 15 million USD, with software, firmware, and hardware developed by multiple independent physics and engineering teams at different colleges and universities.  A multi-institution collaboration is required to complete this scale of project.  This is a standard field-wide in astroparticle physics, as these major science facilities cannot be supported by a single university or national lab.  The detector will be comprised of $\approx$ 100 detector modules, all independently observing $\approx$ 1 km$^3$ of ice.  Antarctic deployments require coordination and planning best acheived by a collaboration rather than a small group of individuals.  In addition to working on the detectors and data analysis, I also participate in RF measurements of polar ice properties, to better understand how the UHE neutrino signals would propagate $\approx 1$ km to our detector modules.  These ice properties include how much energy is lost by the RF pulse in the ice, and how the trajectory of the radio wave is curved due to the changing density of ice versus depth.  I was one of the first researchers in the world to quantify the energy loss of RF pulses for the ice of ARIANNA \cite{hanson2015}.

\subsection{Conferences and Workshops}



\end{document}