\documentclass[../../main.tex]{subfiles}

\begin{document}

Recently, I have facilitated merger discussions between the ARA and ARIANNA collaborations.  National Science Foundation (NSF) program officers have committed to strongly support a brand new neutrino detector in Antarctica if ARA/ARIANNA merge.  The NSF solicitation is entitled ``Windows on the Universe: The Era of Mutli-Messenger Astrophysics''\footnote{See the website \url{https://www.nsf.gov/funding/pgm_summ.jsp?pims_id=505593}.}, and is due December 4th, 2018.  The NSF liason is Vladimir Papitashvili (vpapita@nsf.gov) from the NSF Office of Polar Programs (OPP).  Five years after receiving my PhD, I am proud to state that we are on the verge of major detector construction, due to workshops I helped to organize at both UC Irvine and Ohio State this year.  The timing of this merger is especially impactful for Whittier College undergraduates, for soon the joint ARA/ARIANNA\footnote{We are still in the process of choosing a new name.} collaboration will require a small army of undergraduate researchers. Whittier is ideally located near UC Irvine, a founding institution and the center for RF hardware and firmware development.  More detail about this proposal, including Whittier undergraduate involvement, is included below in Table \ref{}. \\ \hspace{0.1cm}

\begin{table}
\centering
\begin{tabular}{|c|c|c|c|}
\hline
Proposal Funding Level & US Institution & Technical Skills Required & Whittier Undergraduates \\ \hline \hline
\$15 million & \makecell{Univ. of Chicago \\ Univ. of Kansas} & Hardware & AA \\
\end{tabular}
\end{table}

The funding scale for our Antarctic neutrino detector is approximately 15 million USD, with software, firmware, and hardware developed by multiple independent physics and engineering teams at different colleges and universities.  A multi-institution collaboration is required to complete this scale of project.  This is a standard field-wide in astroparticle physics, as these major science facilities cannot be supported by a single university or national lab.  The detector will be comprised of $\approx$ 100 detector modules, all independently observing $\approx$ 1 km$^3$ of ice\footnote{Currently, there are 11 ARIANNA and 6 ARA detector modules deployed.}.  Antarctic deployments require coordination and planning best acheived by a collaboration. \\ \hspace{0.1cm}

My near-term research agenda for ARA/ARIANNA involvement focuses on several areas.  The first is simulation software design, in which my student \textbf{Cassady Smith} and I study how the RF pulse propagation in the ice affects UHE neutrino sensitivity.  The second is detector design, in the form of firmware and hardware development in my RF design and testing lab.  My student \textbf{John-Paul G\'{o}mez-Reed} and I are working on a firmware upgrade without which the ARIANNA/ARA detector array could not expand further.  Third, I participate in RF measurements of polar ice properties, to better understand how RF signals propagate $\approx 1$ km to our detector modules.  I was one of the first researchers in the world to quantify the energy loss of RF pulses for the ice of ARIANNA \cite{hanson2015}. Finally, I have begun an RC drone construction program with my student \textbf{Nicholas Clarizio}.  We have been asked by the collaboration to build drones that can emit RF pulses for detector calibration.  \\ \hspace{0.1cm}

\textbf{The future science goals fall into three categories which are unobtainable by any other current collaborations.}  \textbf{The first line of research is to reveal the origin of cosmic rays, which remains unknown after 100 years.}  UHE neutrinos are predicted to exist up to the same energies as cosmic ray protons, as the protons produce neutrinos when they interact with other particles in deep space.  Thus far, IceCube has only been able to observe neutrinos up to 1/1000th the energy of cosmic ray protons, and it's unclear they are related.  Cosmic rays likely originate in other galaxies, but the initial directions and energies are modified because cosmic rays are charged.  When charged particles pass through magnetic fields, like those created by our sun and our galaxy, the initial trajectory is bent and energy is lost.  Neutrinos travel in straight lines through these magnetic fields because they are not charged.  Thus, UHE neutrinos should reveal the origin of cosmic rays by pointing back to the source of the cosmic rays that created them. \\ \hspace{0.1cm}

\textbf{The second line of research focuses on the UHE neutrinos themselves.}  QFT tells us that the laws of physics governing fundamental particles unify and change at ever higher energies.  One example would be to measure the probability that UHE neutrinos interact in ice, and check for any deviation from standard QFT.  Measuring the relative abundance of the three different types\footnote{The three types are named \textit{flavors}, and electrons also come in three \textit{flavors}.  In normal matter, we are accustomed to dealing with only the first type of electron.} of UHE neutrinos and whether there is a deviation from standard QFT is another example.  My former advisor Amy Connolly (Ohio State) has published recently in this area \cite{bustamante2017measurement} \cite{connolly2011calculation}. \\ \hspace{0.1cm}

\textbf{The third line of future research focuses on the potential to correlate UHE neutrino sources with other UHE astrophysical sites, such as \textit{blazars.}}  A blazar is a high-energy gamma-ray emitting galaxy with the beam of gamma-rays oriented toward the Earth.  Recently, IceCube claimed to observe neutrinos from the direction of a blazar that emitted a flare of gamma rays \cite{eaat2890}.  Although not yet confirmed, this would represent an exciting development in the fields of blazars and my own field.  This development would be serendipitous for Whittier College, given that Dr. Glenn Piner studies blazars \cite{piner2018multiepoch} and I study UHE neutrinos. Thus, there is potential for interdepartmental collaboration. \\ \hspace{0.1cm}

\subsection{Conferences, Workshops and Colloquia}

I have summarized all my research lectures given in my first year as a Whittier Professor in Tab. \ref{tab:conf}, and I have included the undergraduates who helped me with the research.  The purpose of this listing is two-fold: to demonstrate that I have already begun to expose my research students to the broader field of astroparticle physics, and to demonstrate that I am both an active member of the scientific and Whittier communities.\footnote{Student feedback on the Whittier College colloquium is provided in the supplemental material.}

\begin{table}
\small
\centering
\begin{tabular}{|c|c|c|}
\hline \hline
Event & Date/Location & Undergraduates Involved \\ \hline
RF Prop. Coding Workshops & Oct. 2017 at UC Irvine & Accompanied by C. Smith \\ \hline
Invited Colloquium & April 19th, 2018 at Cal Poly San Luis Obispo & Research aided by C. Smith \\ \hline
Department Colloquium & May 4th, 2018 at Whittier College & Research aided by C. Smith \\ \hline
UHE eng. meeting & July 11th, 2018 at UC Irvine & Accompanied by Keck Fellows \\ \hline
UHE eng. meeting & August 29-31, 2018 at UC Irvine & Research aided by J.P. G\'{o}mez-Reed \\ \hline
Merger meeting & September 12-15, 2018 at Ohio State University & Research aided by Keck Fellows \\ \hline
Public lecture & September 26th, 2018 at Los Nietos Middle School & Research aided by Keck Fellows \\ \hline
\hline
\end{tabular}
\caption{\label{tab:conf} Cassady Smith (C. Smith) began performing software development and research with me in Fall 2017, and I took her to meet my colleagues at UC Irvine in Fall 2017. Also in Fall 2017, I was introduced to John-Paul (J.P.) G\'{o}nez-Reed, who began developing firmware in my lab.  I was invited to Cal Poly San Luis Obispo to give a colloquium by Dr. Stephanie Wissel.  I gave a similar colloquium at the end of Spring 2018.  I helped J.P. and Cassady win Keck Summer Research Fellowships, and they both accompanied me to UC Irvine for ARIANNA engineering collaboration.  A second such meeting took place at UCI in late August, beginning the merger discussions.  Finally, I attended the merger meeting between ARA and ARIANNA researchers at Ohio State, where I presented my Keck Fellows' research alongside my own.}
\end{table}

\end{document}