\documentclass[../../main.tex]{subfiles}

\begin{document}

Recently I helped facilitate merger discussions between ARA and ARIANNA.  National Science Foundation (NSF) program officers have committed support for a brand new neutrino detector in Antarctica if ARA/ARIANNA merge.  The NSF solicitation is entitled ``Windows on the Universe: The Era of Mutli-Messenger Astrophysics''\footnote{See the website \url{https://www.nsf.gov/funding/pgm_summ.jsp?pims_id=505593}.}, and is due December 4th, 2018.  The NSF liaison is Vladimir Papitashvili (vpapita@nsf.gov) from the NSF Office of Polar Programs (OPP).  Five years after receiving my PhD, I am proud to state that we are on the verge of major detector construction, due to workshops I helped to organize at both UC Irvine and Ohio State this year. \\ \hspace{0.1cm}

\begin{table}[ht]
\centering
\begin{tabular}{|c|c|c|c|}
\hline
Proposal Funding Level & US Institutions & Technical Areas & Whittier Undergraduates \\ \hline \hline
\$20 million & \makecell{Whittier College \\ UC Irvine \\ California Polytechnic (SLO) \\ Lawrence Berkeley Lab \\ Univ. of Chicago \\ Univ. of Kansas \\ University of Wisconsin \\ Ohio State University \\ University of Delaware \\ Otterbein University} & \makecell{RF Hardware/calibration \\ RF Firmware \\ Ice property measurement \\ Software development \\ Deployment and logistics} & \makecell{John-Paul G\`{o}mez-Reed \\ (RF Firmware) \\ Cassady Smith \\ (Software development) \\ Nicholas Clarizio \\ (Hardware calibration)} \\ \hline
\end{tabular}
\caption{\label{tab:proposal} Overview of the upcoming ARA/ARIANNA proposal.  The overall budget for the proposal will be approximately 20 million USD.  There are currently 10 US institutions, and Whittier College is the only Title V HSI.  If successful, a portion of the grant will go to Whittier College, with myself as a co-PI.  The work required will be in five technical areas.  My student Cassady Smith is helping to develop software that simulates the UHE neutrino interactions in ice.  John-Paul G\`{o}mez-Reed is assisting in developing firmware for the station electronics.  Finally, Nihcolas Clarizio is building a drone to assist with deployed hardware calibration.}
\end{table}

The timing of this merger is especially important for Whittier College undergraduates, for soon the joint ARA/ARIANNA\footnote{We are still in the process of choosing a new name.} collaboration will require a small army of undergraduate researchers.  For more detail about this proposal, see Table \ref{tab:proposal}.  The funding scale for our Antarctic neutrino detector is approximately 20 million USD, with software, firmware, and hardware developed by multiple independent physics and engineering teams at different colleges and universities.  A multi-institution collaboration is required to complete this scale of project.  This is a standard field-wide in astroparticle physics, as these major science facilities cannot be supported by a single university or national lab.  The future goals of the science created by the detector are summarized below. \\ \hspace{0.1cm}

\begin{itemize}
\item \textbf{\textit{Determine the origin of cosmic rays, which remains unknown after 100 years.}}  Cosmic rays likely originate in other galaxies, but the initial directions are modified because cosmic rays are charged.  When charged particles pass through magnetic fields, like those created by our sun and our galaxy, the initial trajectory is lost. The chargeless UHE neutrinos will point back to the source of cosmic rays, and the properties of the \textit{energy spectrum} of the UHE neutrinos will also contain clues.
\item \textbf{\textit{Probe QFT at unreachable energies on Earth.}}  The laws of physics governing fundamental particles unify and change at ever higher energies.  One example is to measure the probability that UHE neutrinos interact quantum mechanically in ice, and check for any deviation from standard QFT. Another example is to measure the relative abundance of the three different quantum flavors of cosmic neutrinos \cite{bustamante2017measurement} \cite{connolly2011calculation}. 
\item \textit{\textbf{Investigate correlations between UHE neutrino sources with other UHE astrophysical sites.}}  Two recent examples are given: \textit{blazars} \cite{eaat2890}.  This development would be serendipitous for Whittier College, given that Dr. Glenn Piner studies blazars \cite{piner2018multiepoch} and I study UHE neutrinos. Thus, there is potential for collaboration.
\end{itemize}

\subsection{Conferences, Workshops and Colloquia}

I have summarized all my research lectures given in my first year as a Whittier Professor in Tab. \ref{tab:conf}, and I have included the undergraduates who helped me with the research.  The purpose of this listing is two-fold: to demonstrate that I have already begun to expose my research students to the broader field of astroparticle physics, and to demonstrate that I am both an active member of the scientific and Whittier communities.\footnote{Student feedback on the Whittier College colloquium is provided in the supplemental material.}

\begin{table}
\small
\centering
\begin{tabular}{|c|c|c|}
\hline \hline
Event & Date/Location & Undergraduates Involved \\ \hline
RF Prop. Coding Workshops & Oct. 2017 at UC Irvine & Accompanied by C. Smith \\ \hline
Invited Colloquium & April 19th, 2018 at Cal Poly San Luis Obispo & Research aided by C. Smith \\ \hline
Department Colloquium & May 4th, 2018 at Whittier College & Research aided by C. Smith \\ \hline
UHE eng. meeting & July 11th, 2018 at UC Irvine & Accompanied by Keck Fellows \\ \hline
UHE eng. meeting & August 29-31, 2018 at UC Irvine & Research aided by J.P. G\'{o}mez-Reed \\ \hline
Merger meeting & September 12-15, 2018 at Ohio State University & Research aided by Keck Fellows \\ \hline
Public lecture & September 26th, 2018 at Los Nietos Middle School & Research aided by Keck Fellows \\ \hline
\hline
\end{tabular}
\caption{\label{tab:conf} Cassady Smith (C. Smith) began performing software development and research with me in Fall 2017, and I took her to meet my colleagues at UC Irvine in Fall 2017. Also in Fall 2017, I was introduced to John-Paul (J.P.) G\'{o}mez-Reed, who began developing firmware in my lab.  I was invited to Cal Poly San Luis Obispo to give a colloquium by Dr. Stephanie Wissel.  I gave a similar colloquium at the end of Spring 2018.  I helped J.P. and Cassady win Keck Summer Research Fellowships, and they both accompanied me to UC Irvine for ARIANNA engineering collaboration.  A second such meeting took place at UCI in late August, beginning the merger discussions.  Finally, I attended the merger meeting between ARA and ARIANNA researchers at Ohio State, where I presented my Keck Fellows' research alongside my own.}
\end{table}

\end{document}