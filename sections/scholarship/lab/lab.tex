\documentclass[../../main.tex]{subfiles}

\begin{document}

Given that I am striving for quality research in \textit{all phases} of my sub-field, I have built an RF circuit fabrication and testing station in my laboratory in the Science and Learning Center.  I have begun to teach myself \textit{firmware programming}, which is a wholly different discipline than \textit{software programming.}  Software code is written in languages like C++ and compiled into assembly and then binary to be executed by a CPU.  Firmware code is written in languages like Verilog and VHDL and is compiled directly to binary, which is translated into a physical circuit on a field-programmable gate array (FPGA).  The combination of firmware code and FPGAs leads to reprogrammable microcircuits that perform repeated digital tasks at $\approx$ GHz clock frequencies.  This discipline is relevant to my research in that the ARIANNA and ARA detector modules make heavy use of FPGAs in their sub-systems to filter and process RF pulse data.  My students and I have already begun to modify and update the ARIANNA firmware on an actual ARIANNA detector module from UC Irvine.  This has given my students hands-on experience with firmware development.  The firmware upgrades we are developing are a vital part of the detector expansion currently underway in the merge of ARA and ARIANNA.

The sensitivity of ARA/ARIANNA to UHE cosmic rays and neutrinos is proportional to the number of RF antenna channels deployed.  Currently, there is a limit on the number of channels we can deploy, caused by man-made and natural RF noise.  Man-made radio noise comes from communications bands in the 100-1000 MHz range, including military satellites \cite{ALLISON201847}.  Examples of natural RF noise are \textit{thermal noise}, in which the electrons radiate at the ice temperatures, leading to our detectors triggering at a random but predictable rate, and the synchrotron electron emissions from the Milky Way \cite{barwick2016radio}.  This firmware upgrade must also adapt to changing wind conditions autonomously, because nearby wind turbines at South Pole emit noise that varies with wind speed. As mentioned above, Whittier 3-2 Math/Computer-Science major J.P. G\'{o}mez-Reed has competed major research milestones in this area as a Keck Fellow.  Now that we have begun to write Verilog and VHDL programs, we can give the ARIANNA detector modules new RF noise-rejecting abilities, thus reducing the overall data rate of the RF channels.  This firmware upgrade is required to give the detector new sensitivity in the years ahead by allowing deployment of more RF channels, and thus more detector modules.  By increasing the number of detector modules, we build up to the necessary 100 km$^3$ of observed ice to record the first UHE neutrino signals. We plan on contracting with ARA/ARIANNA to bring funding to Whittier College as firmware developers in the coming months.

\end{document}