\documentclass[../../main.tex]{subfiles}

\begin{document}

Given that I am striving for quality research in \textit{all phases} of my sub-field, I have built an RF circuit fabrication and testing station in my laboratory in the Science and Learning Center.  I have begun to teach myself \textit{firmware programming}, which is a wholly different discipline than \textit{software programming.}  Software code is written in languages like C++ and compiled into assembly and then binary to be executed by a CPU.  Firmware code is written in languages like Verilog and is compiled directly to binary, which is translated into a physical circuit on a field-programmable gate array (FPGA).  The combination of firmware code and FPGAs leads to reprogrammable microcircuits that perform repeated digital tasks at $\approx$ GHz clock frequencies.  This discipline is relevant to my research in that the ARIANNA and ARA detector modules make heavy use of FPGAs in their sub-systems to filter and process RF pulse data.  My students and I have already begun to modify and update the ARIANNA firmware on an actual ARIANNA module on-loan from UC Irvine.  This has given my students hands-on experience with firmware development.  The firmware upgrades we are developing are a vital part of the detector expansion currently underway in the merge of ARA and ARIANNA.  We plan on contracting with the collaboration to bring funding to Whittier College as a developer for ARIANNA modules in the coming year.  I already have one 3-2 engineering major, John-Paul G\'{o}mez-Reed, working with me on ARIANNA firmware development.  With me as advisor, John-Paul received a Keck Summer Research Fellowship for Summer 2018.

\end{document}