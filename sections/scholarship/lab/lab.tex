\documentclass[../../main.tex]{subfiles}

\begin{document}

\textbf{Given that I am striving for quality research in \textit{all phases} of my sub-field, I have built an RF circuit fabrication and testing laboratory the Science and Learning Center.}  I am grateful to Dean Darrin Good for the startup grant that made this effort possible.  With this facility, I teach my students \textit{firmware programming}.  Software code is written in languages like C++ and compiled to be executed by a CPU.  Firmware code is written in languages like Verilog and VHDL, which is translated into a physical circuit on a field-programmable gate array (FPGA).  FPGA designs are reprogrammable microcircuits that can perform arbitrary digital tasks at high speeds ($\approx$ GHz clock frequencies).  In addition to the curricular implications for my new course, Computer Logic and Digital Circuit Design (COSC330/PHYS306), this lab supports my research with ARA/ARIANNA in several ways, summarized in Tab. \ref{tab:fpga}. \\ \hspace{0.1cm}

\begin{table}[h]
\centering
\begin{tabular}{|c|c|c|}
\hline
\textbf{Project} & \textbf{Scientific purpose} & \textbf{Participants} \\ \hline \hline
Digital trigger throttle & \makecell{Automated measurement of RF trigger rate \\ in ARA/ARIANNA.  Enables detector mass-production via \\ automation of threshold tuning.} & \makecell{J.C. Hanson \\ J.P. G\`{o}mez-Reed} \\ \hline
Digital phased-array & \makecell{Enables lowering RF threshold \\ for recording increasing number of UHE neutrino signals.} &  \makecell{J.C. Hanson} \\ \hline
Majority logic-upgrades & \makecell{Enables use of multiple majorities \\ of different RF antenna types to detect UHE neutrinos} & \makecell{J.C. Hanson \\ J.P. G\`{o}mez-Reed} \\ \hline
Digital RF chirp trigger & \makecell{Allows us to lower thresholds, taking advantage of \\ radio antenna properties} & \makecell{J.C. Hanson} \\ \hline
\hline
\end{tabular}
\caption{\label{tab:fpga} FPGA-based projects in our RF lab.  Each of the projects above contributes to the ARA/ARIANNA \textit{trigger} (see text for details).  The purpose of the digital trigger throttle is to automate RF channel calibration, necessary for a detector built from thousands of channels.  The purpose of the phased array project is to more efficiently use information from all channels in real time.  The third project explores using different antenna subsets for efficient triggers.  The final project takes advantage of the \textit{chirping} property of the RF antennas used in ARIANNA, allowing for lower thresholds for the trigger.  My Keck Fellow J.P. G\`{o}mez-Reed made excellent progress with the first and third projects during Summer 2018.}
\end{table}

The sensitivity of ARA/ARIANNA to UHE neutrinos increases with the number of RF antenna channels deployed.  Each has an RF \textit{threshold} and \textit{trigger}.  The incident RF pulse must have a voltage larger than the threshold for that channel to record data (to trigger).  The lower the threshold, the higher the chances that channel has of observing a UHE neutrino signal\footnote{There are many more UHE neutrino signals that produce relatively small signals, versus large ones, compared to the thermal environment around the antenna.}.  Naively lowering thresholds comes at the price of triggering on thermal noise and man-made signals \cite{ALLISON201847} \cite{barwick2016radio}.  The projects in Tab. \ref{tab:fpga} are all geared towards lowering the threshold while avoiding triggering on unwanted signals.  These operations are deemed mandatory by the ARA/ARIANNA collaboration.  My student, J.P. G\'{o}mez-Reed, has competed major research milestones in the digital trigger throttle project as a Keck Fellow in my lab. \\ \hspace{0.1cm}

Firmware development is also crucial to the education of our STEM curriculum.  \textit{I am the first professor at Whittier College to teach it}, and soon it will become a core component of the computer science curriculum through my new upper division course (COSC330/PHYS306).  I am gradually increasing the firmware component of COSC330.  Our STEM students will be better prepared for research work, or to gradaute from USC through the 3-2 program if they have encountered firmware development, as it is increasingly becoming standard undergraduate engineering curriculum.  for those students considerin careers in science, those who can write firmware applications are of singular importance to collaborations like my own.  I'm grateful to Dean Good for providing me start-up resources so I can begin this tradition at Whittier College.

\end{document}