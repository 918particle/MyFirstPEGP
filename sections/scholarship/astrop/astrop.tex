\documentclass[../../main.tex]{subfiles}

\begin{document}

Astroparticle physics is simply the combination of \textit{astrophysics} and \textit{particle physics.}  The study of astrophysics may be further subdivided into observational astronomy, which is the study of the origin and phenomenology of celestial objects, and the physics of celestial objects themselves.  The former is based on observations from telescopes sentitive to various parts of the electromagnetic spectrum, and the latter involves topics like special relativity, general relativity and cosmology.  Particle physics involves the study of atoms and photons, electrons and nuclei, and the quantum field theory that governs all sub-atomic particles.  For the last 100 years, the main sub-field at the juncture between astrophysics and particle physics was the study of \textit{cosmic rays}.  Cosmic rays are UHE nuclei, photons, electrons, and neutrinos that arrive at Earth from deep space.  Particle physics data suggests that the fundamental forces in nature unify under one quantum field theory as the particle energy increases, and the particle speed approaches the speed of light.  Before humans could accelerate sub-atomic particles to such energies, we studied them exclusively in the form of cosmic rays.  The Nobel prize-winning discoveries of cosmic radiation, antimatter, the cloud chamber, the muon (a heavier version of an electron), quantum neutrino oscillations and solar neutrinos, and gravitational waves have all come from a sub-field that evolved from cosmic-ray physics into what we now call \textit{astroparticle physics}, or \textit{multi-messenger astronomy}. \\ \hspace{0.1cm}

Owing to its remote location and isolation, the excellent transparency of ice at wavelengths ranging from optical through radio, and the presence of extensive scientific support at research bases, \textbf{Antarctica} now supports multiple astronomy and astrophysics-oriented projects.  Within the last five years,  the IceCube experiment, sensitive to optical and near-optical Cherenkov radiation resulting from neutrino interactions in ice, has reported on the first observation of extraterrestrial neutrinos at world-record energies \cite{Aartsen:2016xlq}.  At higher energies than IceCube, in-ice detection of longer-wavelength (radio) radiation is likely a more sensitive detection strategy, owing to the Askaryan effect \cite{Askaryan:1962hbi,1962JPSJS..17C.257A,1965JETP...21..658A}, combined with the observation that RF pulses propagate for kilometers in cold polar ice \cite{barrella_barwick_saltzberg_2011,barwick_besson_gorham_saltzberg_2005}.  The Askaryan effect occurs when a UHE particle interacts in solid matter, creates a cascade of new particles, and radiates RF pulses.  Several pioneering efforts to capture UHE neutrinos from outside our solar system via the Askaryan effect have been undertaken in Antarctica \cite{Allison:2015eky,Barwick:2014pca,Gorham:2008dv,Kravchenko:2001id}.  Two of these are the Askaryan Radio Array (ARA) and the Antarctic Ross Ice Shelf Antenna Neutrino Array (ARIANNA).  My research focuses on combining the best of the two detectors, merging them into one, and achieving world-record breaking UHE neutrino observations a factor of 1000 times more energetic than those of IceCube.

\subsection{Recent Peer-Reviewed Publications, with Brief Descriptions}

Below is a selection of xxx \textit{recent} peer-reviewed publications in this field, in reverse chronological order.  I indicate in the list the papers for which I was the  \textit{corresponding author} with an asterisk (*).  In the fields of particle and astrophysics, the corresponding author is the one responsible for the paper.  We have a policy of alphabetical author listing, in order to recognize the contributions of all team members, which often runs counter to the expectations of researchers in other fields used to reading the ``first author'' as the main contributor.  Recently, I have averaged one peer-reviewed journal article for which I was the corresponding author per year for the last three years.

\begin{enumerate}
\item J.C. Hanson et al. ``Observation of classically ‘forbidden’ electromagnetic wave propagation and implications for neutrino detection.'' Journal of Cosmology and Astroparticle Physics. \textbf{(2018)} (2018).
\begin{itemize}
\item \textit{In my first publication as a Whittier College Assistant professor, we reported the first measurements of horizontal RF propagation in Antarctic Ice.  These observations and their significance to ARA and ARIANNA are described below.}
\item \textit{For technical reasons, I could not be the corresponding author.  However, I wrote one-third of this work, including break-through theoretical calculations that led to a better understanding of the data.}
\item \textit{For the past few years, we have not been holding physics colloquia at Whittier College.  I renewed that tradition by giving a colloquium at the end of the spring semester, 2018.}
\item \textit{My student Cassady Smith was involved with this research, and I helped her to obtain a 2018 Keck summer research fellowship to work on this with me.}
\end{itemize}
\item *J.C. Hanson and A. Connolly. ``Complex Analysis of Askaryan Radiation: A Fully Analytic Treatment including the LPM effect and Cascade Form Factor.'' Astroparticle Physics. \textbf{(91)} pp. 75-89 (2017).
\begin{itemize}
\item \textit{I derived the first complex analytical theory of the Askaryan effect (described below) to include crucial formulations of the LPM effect and the shape of the UHE particle cascade created by a neutrino in ice.}
\item \textit{Complex analytical calculations facilitate high-speed computational simulations of ARA/ARIANNA neutrino detection.}
\item \textit{Prior calculations neglected the LPM effect, which occurs at energies relevant to ARA/ARIANNA.}
\item \textit{I'm grateful to my advisor Amy Connolly for editing the paper, so I gave her authorship credit.  We were finishing the paper as I arrive at Whittier College.}
\end{itemize}
\item The ARIANNA Collaboration. ``Radio detection of air showers with the ARIANNA experiment on the Ross Ice Shelf'', Astroparticle Physics \textbf{(90)} pp. 50-68 (2017).
\begin{itemize}
\item \textit{This is the world's only detection of UHE cosmic rays based exclusively on RF signals from air-showers.  Air-showers are the high-energy cascades left by cosmic rays in the Earth's atmosphere.}
\item \textit{Other experiments rely on high-energy muon signals and record the RF signals as a by-product.}
\item \textit{RF detection of cosmic rays is orders of magnitude cheaper than other methods.}
\item \textit{I was involved in the data analysis of this paper.  By measuring the time-domain ``response'' of the ARIANNA systems (one of my papers below), I was able to discern which RF signals in the data were cosmic rays.  I provided these insights to others who finished the analysis with simulations.}
\end{itemize}
\item The TARA Collaboration. ``First Upper Limits on the Radar Cross Section of Cosmic-Ray Induced Extensive Air Showers'', Astroparticle Physics \textbf{(87)} pp. 1-17 (2017).
\begin{itemize}
\item \textit{In this project, we attempted to measure the radar cross-section of UHE cosmic ray cascades in the Earth's atmosphere. A radar cross-section quantifies how much RF energy is reflected by an object illuminated by radar.}
\item \textit{These cascades are several kilometers long, but it turns out that they are so thin that we could only set an upper limit on the radar cross-section.}
\end{itemize}
\item *J.C. Hanson et al. ``Time-Domain Response of the ARIANNA Detector.'' Astroparticle Physics \textbf{(62)} pp. 139-151 (2015).
\begin{itemize}
\item \textit{This paper provided key details to suppor the conclusions of other major recent publications, including the radio detection of air showers and the first-search for cosmogenic neutrinos with ARIANNA.}
\item \textit{These measurements, the first of their kind, were taken at the University of Kansas.}
\end{itemize}
\item *J.C. Hanson et al. ``Radio-frequency Attenuation Length, Basal Reflectivity, Depth, and Polarization Measurements from Moore's Bay in the Ross Ice-Shelf.'' Journal of Glaciology \textbf{(61)} 227, pp. 438-446(9).
\begin{itemize}
\item \textit{Similar to the above paper, this paper provided key details about the ice to the first-search paper below.}
\item \textit{The ice properties affect the sensitivity of our projects to UHE neutrinos.}
\end{itemize}
\item The ARIANNA Collaboration. ``A First Search for Cosmogenic Neutrinos with the ARIANNA Hexagonal Radio Array.'' Astroparticle Physics \textbf{(70)} pp. 12-36 (2015).
\begin{itemize}
\item \textit{This work published the first upper limits on the UHE neutrino flux from outside the Milky Way by ARIANNA.  This paper was the culmination of years of planning, construction, and analysis.}
\item \textit{I helped to deploy the hardware used to collect the data analyzed for this paper.}
\item \textit{Given our current projects, this paper shows that ARA/ARIANNA will be the most sensitive detector in history to UHE neutrinos if built on a mass scale.}
\end{itemize}
\end{enumerate}

\end{document}
