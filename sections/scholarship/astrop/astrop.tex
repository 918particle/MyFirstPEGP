\documentclass[../../main.tex]{subfiles}

\begin{document}

Astroparticle physics is simply the combination of \textit{astrophysics} and \textit{particle physics.}  The study of astrophysics may be further subdivided into observational astronomy, which is the study of the origin and phenomenology of celestial objects, and the physics of celestial objects themselves.  The former is based on observations from telescopes sensitive to various parts of the electromagnetic spectrum, and the latter involves topics like special relativity, general relativity and cosmology.  Particle physics involves the study of atoms and photons, electrons and nuclei, and the quantum field theory (QFT) that governs all sub-atomic particles.  Beginning with formative discoveries in 1911, the main sub-field at the juncture between astrophysics and particle physics has been the study of \textit{cosmic rays}. \\ \hspace{0.1cm}

Cosmic rays are UHE nuclei, photons, electrons, and neutrinos that arrive at Earth from deep space.  Particle physics data suggests that the fundamental forces in nature unify under one QFT as the particle energy increases, and the particle speed approaches the speed of light.  Before humans could accelerate sub-atomic particles to such energies, we studied them exclusively in the form of cosmic rays.  The Nobel prize-winning discoveries of cosmic radiation, antimatter, the cloud chamber, the muon (a heavier version of an electron), quantum neutrino oscillations and solar neutrinos, and gravitational waves have all come from a sub-field that evolved from cosmic-ray physics into what we now call \textit{astroparticle physics}, or \textit{multi-messenger astronomy}. \\ \hspace{0.1cm}

To aid readers outside the field of physics, below is a list of relevant definitions used in this chapter.
\begin{itemize}
\item \textit{Quantum field theory (QFT)} ... Application of quantum mechanics to sub-atomic particles, describing fundamental interactions of matter and light.
\item \textit{Cosmic ray} ... A high-energy proton or other atomic nucleus moving near the speed of light in deep space.
\item \textit{Neutrino} ... A sub-atomic particle emitted in various nuclear and high-energy quantum interactions.
\item \textit{Cherenkov radiation} ... UV light emitted when high-energy charged particles move through ice near the speed of light (or other material with an \textit{index of refraction}.
\item \textit{Index of refraction} ... A number $n$ that quantifies the speed of light in a material: $v = c/n$, where $v$ is the speed, and $c$ is the speed of light in a vacuum.
\item \textit{Askaryan effect} ... This effect occurs when a high-energy particle interacts in a material with an index of refraction, depositing energy in the form of many charged particles.  The charged particles undergo the Cherenkov effect, but radiate together as a group, leading to a radio pulse. 
\item \textit{The Standard Model} ... A QFT model that explains all known sub-atomic phenomena with striking precision, up to a certain energy.  The model predicts quantum effects that explain the behavior of atoms, molecules, chemistry, etc.  Although the model is powerful, neutrinos have properties not predicted by it.
\item \textit{ARIANNA} ... Acronym for Antarctic Ross Ice Shelf Antenna Neutrino Array.
\item \textit{ARA} ... Acronymn for Askaryan Radio Array.
\end{itemize}

Owing to its remote location and isolation, the excellent optical and radio transparency of ice, and the presence of extensive scientific support at research bases, \textbf{Antarctica} now supports many astroparticle physics projects.  Within the last five years,  the IceCube experiment, sensitive to Cherenkov radiation resulting from neutrino interactions in ice, has reported on the first observation of extraterrestrial neutrinos at world-record energies \cite{Aartsen:2016xlq}.  In-ice detection of RF radiation, as opposed to Cherenkov radiation, is a more efficient strategy at neutrino energies 1000 times higher than those observed at IceCube. This strategy arrises from the combination of the Askaryan effect \cite{Askaryan:1962hbi,1962JPSJS..17C.257A,1965JETP...21..658A} and the observation that RF pulses propagate for kilometers in ice \cite{barrella_barwick_saltzberg_2011,barwick_besson_gorham_saltzberg_2005}.  The Askaryan effect occurs when a UHE particle interacts in solid matter and radiates an RF pulse.  Pioneering efforts to capture UHE neutrinos originating from beyond the solar system via the Askaryan effect have been undertaken in Antarctica \cite{Allison:2015eky,Barwick:2014pca,Gorham:2008dv,Kravchenko:2001id}.  Two of these are ARA and ARIANNA.  My research focuses on combining the best of these two designs, and achieving world-record breaking UHE neutrino observations. \\ \hspace{0.1cm}

The detection of neutrinos from beyond the solar system at the highest cosmic-ray energies would be a watershed moment in physics for three reasons.  It would aid in the discovery of the source or sources of cosmic rays, which after 100 years remains unknown.  Second, it would represent an interaction with matter that is almost certainly from other galaxies, given that neutrinos propagate through deep space for distances much larger than the Milky Way.  The long distance propagation stems from the fact that neutrinos interact with matter rarely, and because inter-galactic space is filled with very little matter.  Finally, it would allow us to perform fundamental tests of the QFT that forms the Standard Model, which is the global physics model describing all of nature (sub-atomic physics, atomic physics, chemistry, etc.), except gravity.

\subsection{Recent Peer-Reviewed Publications, with Brief Descriptions}

Below is a selection of seven \textit{recent} peer-reviewed publications in this field, in reverse chronological order.  I indicate in the list the papers for which I was the  \textit{corresponding author} with an asterisk (*).  In the fields of particle and astrophysics, the corresponding author is the author who officially submits the work for publication and corresponds with the journal editor.  We also have a policy of strict alphabetical author listing, in order to recognize the contributions of all team members, even if they did not contribute to a particular paper.  This style often runs counter to the expectations of researchers in other fields accustomed to reading the ``first author'' as the main contributor.  I have averaged one peer-reviewed journal article for which I was the corresponding author per year since I graduated from UCI in 2013. \\ \hspace{0.1cm}

The purpose of the list is two-fold.  First, I must demonstrate that I am committed to research excellence and that my research reaches out from Whittier College to the world of astroparticle physics.  \textit{The paper listed in bold face I count towards satisfaction of my departmental tenure guidelines.}  Second, I provide the list as evidence that I have begun to involve Whittier undergraduates in my research program. \\ \hspace{0.1cm}

\begin{enumerate}
\item \textbf{*J.C. Hanson et al. ``Observation of classically ‘forbidden’ electromagnetic wave propagation and implications for neutrino detection.'' Journal of Cosmology and Astroparticle Physics. \textbf{(2018)} (2018).}
\begin{itemize}
\item \textit{In my first publication as a Whittier College Assistant professor, we reported the first measurements of horizontal RF propagation in Antarctic Ice.  The data was collected over many years, but I performed all of my work for this paper while at Whittier College.  These observations and their significance to ARA and ARIANNA are described below.}
\item \textit{I am a corresponding author on this paper, and I wrote one-third of this work.  My contributions include collecting a large fraction of the data, and break-through theoretical calculations that led to a better understanding of the data.}
\item \textit{For the past few years, we have not been holding physics colloquia at Whittier College.  I renewed that tradition by giving a colloquium at the end of Spring 2018 on this research.}
\item \textit{\textbf{My student Cassady Smith was involved with this research, and I helped her to obtain a 2018 Keck summer research fellowship to work on this with me.}}
\end{itemize}
\item *J.C. Hanson and A. Connolly. ``Complex Analysis of Askaryan Radiation: A Fully Analytic Treatment including the LPM effect and Cascade Form Factor.'' Astroparticle Physics. \textbf{(91)} pp. 75-89 (2017).
\item The ARIANNA Collaboration. ``Radio detection of air showers with the ARIANNA experiment on the Ross Ice Shelf'', Astroparticle Physics \textbf{(90)} pp. 50-68 (2017).
\item The TARA Collaboration. ``First Upper Limits on the Radar Cross Section of Cosmic-Ray Induced Extensive Air Showers'', Astroparticle Physics \textbf{(87)} pp. 1-17 (2017).
\item *J.C. Hanson et al. ``Time-Domain Response of the ARIANNA Detector.'' Astroparticle Physics \textbf{(62)} pp. 139-151 (2015).
\item *J.C. Hanson et al. ``Radio-frequency Attenuation Length, Basal Reflectivity, Depth, and Polarization Measurements from Moore's Bay in the Ross Ice-Shelf.'' Journal of Glaciology \textbf{(61)} 227, pp. 438-446(9).
\item The ARIANNA Collaboration. ``A First Search for Cosmogenic Neutrinos with the ARIANNA Hexagonal Radio Array.'' Astroparticle Physics \textbf{(70)} pp. 12-36 (2015).
\end{enumerate}

The Department of Physics and Astronomy at Whittier College has clearly defined requirements for granting tenure and promotions.  In Tab. \ref{tab:tenure}, the departmental criteria for faculty seeking tenure are summarized \footnote{The departmental guidelines are included in the supporting materials.}.  I have already begun publishing scientific research as a Whittier College, with one peer-reviewed scientific journal article in my first year as a faculty member.  In the next section I outline my plans for grant applications, and future research.

\begin{table}
\centering
\begin{tabular}{|c|c|c|}
\hline \hline
Category & Requirement & Promotion level \\ \hline
\textbf{Teaching} & Teach full course load & Associate Professor \\ \hline
\textbf{Teaching} & Contribute to major and liberal arts curriculum & Associate Professor \\ \hline
\textbf{Teaching} & \makecell{Establish excellent teaching practices, \\ as evaluated and judged by students and colleagues} & Associate Professor \\ \hline
\textbf{Advising} & Encouraged to mentor at least one cohort of students & Associate Professor \\ \hline
\textbf{Service} & Encouraged to participate on major committees & Associate Professor \\ \hline
\textbf{Scholarship} & Three \textit{peer-reviewed products} & Associate Professor \\ \hline \hline
\textbf{All} & Differ to FPC, Candidate chooses Boyer categories & Full Professor \\ \hline
\end{tabular}
\caption{\label{tab:tenure} Summary of Physics and Astronomy Department guidelines for receiving tenure.  For receiving tenure at the Associate level, the department defines \textit{peer-reviewed products} as: (a) three scientific journal articles in the research area of the candidate, or (b) two such articles and one external grant for which the the candidate is a major contributor.  With department approval, the following alternate paths are accepted: (c) one scientific journal article in the research area of the candidate and two journal articles on pedagogy and physics-education topics, plus one grant as in (b), or (d) two scientific journal articles in the research area of the candidate and two journal articles on pedagogy and physics-education topics.  The scientific articles should fall under the \textit{scholarship of discovery}, using the Boyer categorization.}
\end{table}
\end{document}
