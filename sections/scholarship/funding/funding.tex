\documentclass[../../main.tex]{subfiles}

\begin{document}

\begin{table}
\small
\centering
\begin{tabular}{|c|c|c|c|}
\hline \hline
Grant Name & Time & Amount & Purpose \\ \hline
Start-up grant & Sept. 2017 & \$30,000 & Creation of RF lab, new curriculum \\ \hline
Faculty Development Grant & Nov. 2017 & \$1500 & Attendance of AAPT Conf \\ \hline
Send, Learn, Return Grant & Nov. 2017 & \$500 & Air-travel to AAPT (above). \\ \hline
DigLibArts Grant & Dec. 2017 & \$2000.0 & Development of The Primer \\ \hline \hline
Cotrell Scholars Program & Third year at Whittier & \$100,000 & Antarctic deployments \\ \hline
NSF CAREER Grant & Third year at Whittier & \$100,000-\$500,000 & Completing RF design lab \\ \hline
\hline
\end{tabular}
\caption{\label{tab:fund} A summary of funding for research projects already received, and (below double line) future planned grant applications.  Projects above the double line help build toward successful applications below the double line. Internal Whittier College grants (Faculty Development and Send, Learn, Return) helped fund attendance to the American Association of Physics Teachers Conference for New Professors.  The DigLibArts grant has helped begin The Primer project.  I plan on attempting each of the two grants below the double line within the next two years.  For the NSF CAREER grant, my sub-field of UHE neutrinos has had a nice track record, with three awards in the last decade.}
\end{table}

In Tab. \ref{tab:fund}, I outline the grants I have already received\footnote{The AAPT conference described in the table is listed here: \url{http://www.aapt.org/Conferences/newfaculty/nfw.cfm}}, and plan to submit.  The purpose of this listing is to demonstrate my drive to build an externally funded NSF-based research program.  I have the advantage of being at Whittier College, for which the NSF gives special grant-writing status to detail how our research has ``broader impacts'' for the surrounding community.  Projects such as The Primer, and participation in the Artemis program I've carefully chosen to demonstrate to the NSF my willingness to make an impact on the community.  Thus far, I have received internal grants from Whittier College.  However, these contributions to my startup funding as a professor help me seed projects that ultimately benefit the college, in that I can use the progress to build towards larger proposals.  \\ \hspace{0.1cm}

\textbf{When I joined Whittier College, I was awarded a start-up grant of \$30,000, and I am grateful for the opportunity to use these resources to expand our STEM abilities at Whittier for educational and research purposes.}  I am the first professor to teach firmware here, and my students have begun to learn this crucial engineering skill.  We have used these resources to create a lab which has made progress for ARA/ARIANNA, and helped to place Whittier College firmly within a new field of research.

\begin{table}
\small
\centering
\begin{tabular}{|c|c|c|}
\hline \hline
Equipment & Cost & Purpose \\ \hline
Mixed Domain Oscilloscope (200 MHz bandwidth) & \$6300.00 & Multi-purpose RF component analysis and testing \\ \hline
Xilinx Spartan-6 Embedded Kit & \$830.00 & Development platform for Spartan-6 FPGA \\ \hline
ASUS Desktop PC & \$800.00 & Workstation for Xilinx firmware development software \\ \hline
ASUS 1080p PC Monitor & \$120.00 & Screen for PC workstation \\ \hline
Logitech Mouse/Keyboard combo & \$20.00 & For PC workstation \\ \hline
LogicWorks Software & \$70.00 & Educational simulation software for digital circuits \\ \hline
Total & \$8140.00 & RF Lab with FPGA design and testing capability \\ \hline
\hline
\end{tabular}
\caption{\label{tab:fund2} The startup grant awarded to me upon joining the Whittier community has been put to excellent use.  The FPGA-based projects are made possible by the mixed-domain oscilloscope (meaning digital and analogue signals) and the Xilinx hardware and software.  As mentioned above, these projects are important for creating UHE neutrino detectors build with Xilinx firmware.  The MDO bandwidth is restricted to 200 MHz, but \textit{upgradable} to 1GHz if it becomes necessary scientifically.  I felt it was an economical first step, as 1GHz units can cost up to \$20,000.}
\end{table}

\end{document}