\documentclass[../../main.tex]{subfiles}

\begin{document}

I began my professional career as a physicist when I entered graduate school in 2007 at the University of California at Irvine.  UC Irvine was the home of Frederick Reines, the Nobel Laureate who discovered the neutrino along with Clyde Cowan.  UC Irvine has a long tradition of neutrino and particle physics research, participating in the Nobel-winning Super-Kamiokande experiment\footnote{See \url{http://www-sk.icrr.u-tokyo.ac.jp/sk/index-e.html}}, and the Nobel-winning ATLAS and CMS collaborations at the Large Hadron Collider at CERN\footnote{See \url{https://home.cern/}}.  My PhD advisor was Dr. Steve Barwick, who helped to found the Antarctic Ross Ice Shelf Antenna Neutrino Array (ARIANNA) project (see below), an experiment searching for ultra high-energy (UHE) neutrinos from outside the solar system.  I became an expert in analysis of pulsed radio-frequency (RF) data, RF antenna and circuit design, digital signal processing, and the corresponding software development.  \\ \hspace{0.1cm}

After I received my doctorate, I was hired as a post-doctoral fellow at the University of Kansas, where I continued to work on ARIANNA and other RF-based projects.  Finally, in an attempt to bridge the divide between two competing groups within my sub-field of \textit{astroparticle physics}, I applied for and received a fellowship from the Center for Cosmology and Astroparticle Physics (CCAPP) at Ohio State University.  CCAPP is the home of the Askaryan Radio Array (ARA), a competing project to ARIANNA.  One major theme of my publications and projects these past few years has been to build connections between these two groups, in the hope of building a final version of both projects capable of making record-breaking observations of UHE neutrinos.  I have published in all phases of my sub-field: theoretical calculations, computer simulations, hardware design and deployment, and data analysis\footnote{My C.V. has been included in the supplemental material.}.

\subsection{History of Undergraduate Involvement and Public Engagement}

As a graduate student and post-doctoral fellow, I worked closely with undergraduates to help them understand and participate in the research.  At the University of Kansas, I worked on the QuarkNet program with my advisor Dr. Dave Besson \footnote{https://quarknet.org/}.  The QuarkNet program recruits promising young high-school STEM students to work in a university physics lab over a summer.  I volunteered to aid Dr. Besson with two young women who were making measurements of RF surface waves across materials with a variety of speeds of light.  The students and I were attempting to show that RF surface waves could exist in Antartica, which is relevant to the research I describe below.  As a post-doctoral fellow at CCAPP, I volunteered with my advisor Dr. Amy Connolly on the ASPIRE program, which she created to inspire female high school students to explore careers in STEM.  The program was one week each summer, plus time to recruit the students at local schools.  We gave them projects involving coding, both for computers and Arduino circuit boards \footnote{Arduino is an ubiquitous open-source microcontroller company that supplies DIY boards for prototype designs.}  Also at Ohio State, I gave public lectures to the Columbus Astronomical Society, which is a tradition I am continuing in the Whittier Community by giving lectures at local middle schools.  My first such lecture is this semester at Los Nietos Middle School.  It is my  hope that this history of community engagement aligns with the goals and ideals of Whittier College.

\end{document}
