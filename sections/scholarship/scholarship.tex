\documentclass[../main.tex]{subfiles}
 
\begin{document}

\textbf{Professional Background} \\ \hspace{0.1cm}

I began my professional career as a physicist when I entered graduate school in 2007 at the University of California at Irvine.  UC Irvine was the home of Frederick Reines, the Nobel Laureate who discovered the neutrino along with Clyde Cowan.  UC Irvine has a long tradition of neutrino and particle physics research, participating in the Nobel-winning Super-Kamiokande experiment\footnote{See \url{http://www-sk.icrr.u-tokyo.ac.jp/sk/index-e.html}}, and the Nobel-winning ATLAS and CMS collaborations at the Large Hadron Collider at CERN\footnote{See \url{https://home.cern/}}.  My PhD advisor was Dr. Steve Barwick, who helped to found the Antarctic Ross Ice Shelf Antenna Neutrino Array (ARIANNA) project (see below), an experiment searching for ultra high-energy (UHE) neutrinos from outside the solar system.  I became an expert in analysis of pulsed radio-frequency (RF) data, RF antenna and circuit design, digital signal processing, and the corresponding software development.  \\ \hspace{0.1cm}

After I received my doctorate, I was hired as a post-doctoral fellow at the University of Kansas, where I continued to work on ARIANNA and other RF-based projects.  Finally, in an attempt to bridge the divide between two competing groups within my sub-field of \textit{astroparticle physics}, I applied for and received a fellowship from the Center for Cosmology and Astroparticle Physics (CCAPP) at Ohio State University.  CCAPP is the home of the Askaryan Radio Array (ARA), a competing project to ARIANNA.  One major theme of my publications and projects these past few years has been to build connections between these two groups, in the hope of building a final version of both projects capable of making record-breaking observations of UHE neutrinos.  I have published in all phases of my sub-field: theoretical calculations, computer simulations, hardware design and deployment, and data analysis. \\ \hspace{0.1cm}
 
\textbf{What is Astroparticle Physics?} \\ \hspace{0.1cm}

Astroparticle physics is simply the combination of \textit{astrophysics} and \textit{particle physics.}  The study of astrophysics may be further subdivided into observational astronomy, which is the study of the origin and phenomenology of celestial objects, and the physics of celestial objects themselves.  The former is based on observations from telescopes sentitive to various parts of the electromagnetic spectrum, and the latter involves topics like special relativity, general relativity and cosmology.  Particle physics involves the study of atoms and photons, electrons and nuclei, and the quantum field theory that governs all sub-atomic particles.  For the last 100 years, the main sub-field at the juncture between astrophysics and particle physics was the study of \textit{cosmic rays}.  Cosmic rays are UHE nuclei, photons, electrons, and neutrinos that arrive at Earth from deep space.  Particle physics data suggests that the fundamental forces in nature unify under one quantum field theory as the particle energy increases, and the particle speed approaches the speed of light.  Before humans could accelerate sub-atomic particles to such energies, we studied them exclusively in the form of cosmic rays.  The Nobel prize-winning discoveries of cosmic radiation, antimatter, the cloud chamber, the muon (a heavier version of an electron), quantum neutrino oscillations and solar neutrinos, and gravitational waves have all come from a sub-field that evolved from cosmic-ray physics into what we now call \textit{astroparticle physics}, or \textit{multi-messenger astronomy}. \\ \hspace{0.1cm}

Owing to its remote location and isolation, the excellent transparency of ice at wavelengths ranging from optical through radio, and the presence of extensive scientific support at research bases, \textbf{Antarctica} now supports multiple astronomy and astrophysics-oriented projects.  Within the last five years,  the IceCube experiment, sensitive to optical and near-optical Cherenkov radiation resulting from neutrino interactions in ice, has reported on the first observation of extraterrestrial neutrinos at world-record energies \cite{Aartsen:2016xlq}.  At higher energies than IceCube, in-ice detection of longer-wavelength (radio) radiation is likely a more sensitive detection strategy, owing to the Askaryan effect \cite{Askaryan:1962hbi,1962JPSJS..17C.257A,1965JETP...21..658A}, combined with the observation that RF pulses propagate for kilometers in cold polar ice \cite{barrella_barwick_saltzberg_2011,barwick_besson_gorham_saltzberg_2005}.  The Askaryan effect occurs when a UHE particle interacts in solid matter and radiates RF pulses.  Several pioneering efforts to capture UHE neutrinos from outside our solar system via the Askaryan effect have been undertaken in Antarctica \cite{Allison:2015eky,Barwick:2014pca,Gorham:2008dv,Kravchenko:2001id}.  Two of these are the Askaryan Radio Array (ARA) and the Antarctic Ross Ice Shelf Antenna Neutrino Array (ARIANNA).  My research focuses on combining the best of the two detectors, merging them into one, and achieving world-record breaking UHE neutrino observations a factor of 1000 times more energetic than those of IceCube. \\ \hspace{0.1cm}

\textbf{Antartic Ice as a Particle Detection Medium, and the Future of UHE Neutrino Science} \\ \hspace{0.1cm}

Recently, I have facilitated meetings between the ARA and ARIANNA collaborations, leading to a discussion of a merger.  Program officers from the National Science Foundation have indicated that if the ARA and ARIANNA were to merge, they would strongly support a brand new neutrino detector in Antarctica.  The NSF solicitation is entitled ``Windows on the Universe: The Era of Mutli-Messenger Astrophysics''\footnote{See the website \url{https://www.nsf.gov/funding/pgm_summ.jsp?pims_id=505593}.} has been set for submission of a joint ARA and ARIANNA proposal due December 4th, 2018, with Vladimir Papitashvili (vpapita@nsf.gov) as NSF Office of Polar Programs (OPP) liason.  Five years after receiving my PhD, I am proud to state that we are on the verge of major detector construction due to workshops I helped to organize at both UC Irvine and Ohio State this year.  The timing of this merger is especially impactful for Whittier College, for soon the joint ARA and ARIANNA\footnote{Given that physics professors are not known for their ability to compromise, we are still in the process of choosing a new name.} collaboration will require a small army of undergraduate researchers. We are ideally located near UC Irvine, a founding institution and the center for RF hardware and firmware development. \\ \hspace{0.1cm}

The funding scale for our Antarctic neutrino detector is approximately 15 million USD, with software, firmware, and hardware developed by multiple independent physics and engineering teams at different colleges and universities.  A multi-institution collaboration is required to complete this scale of project.  This is a standard field-wide in astroparticle physics, as these major science facilities cannot be supported by a single university or national lab.  The detector will be comprised of $\approx$ 100 detector modules, all independently observing $\approx$ 1 km$^3$ of ice.  Antarctic deployments require coordination and planning best acheived by a collaboration rather than a small group of individuals.  In addition to working on the detectors and data analysis, I also participate in RF measurements of polar ice properties, to better understand how the UHE neutrino signals would propagate $\approx 1$ km to our detector modules.  These ice properties include how much energy is lost by the RF pulse in the ice, and how the trajectory of the radio wave is curved due to the changing density of ice versus depth.  I was one of the first researchers in the world to quantify the energy loss of RF pulses for the ice of ARIANNA \cite{hanson2015}. \\ \hspace{0.1cm}

\textbf{RF Pulse Propagation in Ice} \\ \hspace{0.1cm}

Given that my sub-field is one collaboration, we publish papers as a collaboration.  In my first publication as a Whittier College professor we reported on the first-ever observations of horizontally propagating RF pulses in polar ice.  Normally, RF pulses travel along curved paths in ice because the speed of light is correlated with density, which changes with depth.  Because the RF pulses are inherently wave-like, \textit{classical} electromagnetic theory states that the RF pulses should swerve downward, because the top of the wave in faster ice outpaces the bottom in the slower ice.  This ``shadowing effect'' is similar to the physics of a mirage, and removes some ice volume from which neutrinos can geometrically originate while still being detectable.  \textbf{In stark contradiction with \textit{classical} propagation theory, we have observed RF pulses traveling in horizontal paths with no apparent curvature, in multiple venues around Antarctica \cite{horizPaper}}.  On a publication with 19 authors, I wrote the first third of this work.  I showed that if we account for density perturbations in the ice, classical horizontal RF propagation is possible.  I have also shown that internal reflection layers (independently observed via radar and field samples) could also lead to horizontally propagation.  \textit{If true, the existence of horizontal RF propagation could increase the probability of UHE neutrino detection} with ARA and ARIANNA, for the simple reason that a larger volume of ice per detector module becomes a potential UHE neutrino target. I am considered a secondary author on these works. I already have one physics major, Cassady Smith, working with me on simulations that quantify these effects.  With me as advisor, Cassady received a Keck Summer Research Fellowship for Summer 2018.  I was a secondary author on two other peer-reviewed works this year: one paper quantified RF propagation measurements at the South Pole with ARA, and the other detailed observations of a solar flare with ARA. \\ \hspace{0.1cm}

\textbf{RF Circuit Fabrication and Testing Laboratory} \\ \hspace{0.1cm}

Given that I am striving for quality research in \textit{all phases} of my sub-field, I have built an RF circuit fabrication and testing station in my laboratory in the Science and Learning Center.  I have begun to teach myself \textit{firmware programming}, which is a wholly different discipline than \textit{software programming.}  Software code is written in languages like C++ and compiled into assembly and then binary to be executed by a CPU.  Firmware code is written in languages like Verilog and is compiled directly to binary, which is translated into a physical circuit on a field-programmable gate array (FPGA).  The combination of firmware code and FPGAs leads to reprogrammable microcircuits that perform repeated digital tasks at $\approx$ GHz clock frequencies.  This discipline is relevant to my research in that the ARIANNA and ARA detector modules make heavy use of FPGAs in their sub-systems to filter and process RF pulse data.  My students and I have already begun to modify and update the ARIANNA firmware on an actual ARIANNA module on-loan from UC Irvine.  This has given my students hands-on experience with firmware development.  The firmware upgrades we are developing are a vital part of the detector expansion currently underway in the merge of ARA and ARIANNA.  We plan on contracting with the collaboration to bring funding to Whittier College as a developer for ARIANNA modules in the coming year.  I already have one 3-2 engineering major, John-Paul G\'{o}mez-Reed, working with me on ARIANNA firmware development.  With me as advisor, John-Paul received a Keck Summer Research Fellowship for Summer 2018.\\ \hspace{0.1cm}

\textbf{Digtal Storytelling and Physics: The Primer} \\ \hspace{0.1cm}

I remember exactly what I thought, the moment I learned of the DigLibArts group at Whittier, and their capabilities.  I thought to myself ``I know exactly what I will do with these capabilities.''  The DigLibArts group introduced me to the Scalar project, which is a digital storytelling tool used to make digital textbooks and other scholarly materials in humanities areas.  I realized that a) there are currently no STEM Scalar projects, and b) this tool could be used to produce a digital physics learning module that employs machine-learning.  I haved connected with WSP and digital media majors, Amy Trinh and Brienne Estrada, who have helped design Adobe-based digital characters for this tool.  Cassady Smith has worked on adding machine-learning algorithms to the project, such that when the tool gathers data on how students proceed through the problems, we gain machine-learning based insight on their learning habits.  I have received a grant from the DigLibArts group to aid in developing this project.  We have named this project The Primer, after 19th Century finishing-school texts meant to educate young ladies.  The choice is in reference to a digital object called "the primer" in a novel entitled ``The Diamond Age,'' by Neal Stephenson, in which a young girl named Nell becomes a world-class engineer when the primer falls into her hands.  I had originally contracted with reknowned game designer and graphic artist \textit{Eric Torres}\footnote{\url{https://ericimagines.com.}} to help us with the designs, but he has had to withdraw for family reasons.  It was a wonderful experience for Amy, Brienne, and Cassady to Skype with him and hear his ideas, however.

\end{document}
