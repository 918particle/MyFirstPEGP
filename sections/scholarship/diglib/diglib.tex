\documentclass[../../main.tex]{subfiles}

\begin{document}

The moment I learned of the DigLibArts group at Whittier, I had the following thought ``I know exactly what I will do.''  The DigLibArts group introduced me to the Scalar project, which is a digital storytelling tool used to make digital scholarly works.  There are currently no STEM Scalar projects.  With this tool, I can produce an interactive digital physics book.  Such a digital text would collect student learning pattern data, and the data could be analyzed with machine-learning techniques.  I haved connected with Whittier Scholars Program (WSP) and digital media majors Amy Trinh and Brienne Estrada, who have helped design Adobe-based digital characters for this tool.  I have received a grant from the DigLibArts group to aid in developing this project.  We have named this project \textit{The Primer}, after 19th Century finishing-school texts.  The choice is in reference to a tool called "The Primer" in a novel entitled ``The Diamond Age,'' by Neal Stephenson, in which a young girl named Nell becomes a world-class engineer when the primer falls into her hands.  I had originally contracted with reknowned game designer and graphic artist \textit{Eric Torres}\footnote{\url{https://ericimagines.com.}} to help us with the designs, but he has had to withdraw for family reasons.  Although losing Eric was a setback, it was a wonderful experience for Amy, Brienne, and Cassady to Skype with him and hear his ideas. \\ \hspace{0.1cm}

The Primer project has also been an excellent opportunity for Cassady Smith to learn about machine learning via the scikit-learn python module\footnote{http://scikit-learn.org/stable/}.  Cassady Smith has been working on python code that will gather data on how students proceed through the physics problems, and will analyze the outcomes via machine-learning algorithms.  Machine-learning algorithms evolve themselves based on training data, solving specific classes of problems efficiently.  We train the algorithms by providing two items.  First, we provide data classifiers, such as the probability students answer a particular question correctly.  Second, we provide the real student outcome, such as the student score on a particular chapter.  I plan to continue The Primer project through recruitment of WSP students, who have the ideal mixed skillset for such a project.  Once the basic code is written and paired with the artwork of Amy Trinh and Brienne Estrada, I will begin sharing it with high-schoolers from the Artemis Program, who will share the Primer with their friends and use it to collect data on how people learn science.  The Artemis program is a program sponsored by the Whittier College Center for Engagement with Communities (CEC) that seeks to empower young women in STEM.  I have volunteered this year to be an Artemis professor coordinator.  This project combines game-like python code editing with data collection and analysis, similar to a physics education research project I encountered at Ohio State as a post-doctoral fellow \cite{orban2017a}.

\end{document}