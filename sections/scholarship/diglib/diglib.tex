\documentclass[../../main.tex]{subfiles}

\begin{document}

The moment I learned of the DigLibArts group at Whittier, I had the following thought ``I know exactly what I will do.''  The DigLibArts group introduced me to the Scalar project, which is a digital storytelling tool used to make digital scholarly works.  There are currently no STEM Scalar projects, but game-like digital physics education projects have been researched\cite{orban2017a}.  I can produce an interactive digital physics book with Scalar.  The book could then collect student learning data, which could be analyzed with machine-learning techniques.  I haved connected with Whittier Scholars Program (WSP) and digital media majors Amy Trinh and Brienne Estrada, who have helped design Adobe-based digital characters for this tool. \\ \hspace{0.1cm}

I have received a grant from the DigLibArts group to aid in developing this project.  We have named this project \textit{The Primer}, after 19th Century finishing-school texts.  The choice is in reference to a tool called ``The Primer'' in a novel entitled ``The Diamond Age,'' by Neal Stephenson.  A young female character named Nell becomes a world-class engineer when The Primer falls into her hands.  I began collaborating with reknowned game designer and graphic artist \textit{Eric Torres}\footnote{\url{https://ericimagines.com.}}, but he  had to withdraw for family reasons.  It was a wonderful experience for Amy, Brienne, and Cassady to Skype with him and hear his ideas. The Primer project has also been an excellent opportunity for Cassady Smith to learn about machine learning via the scikit-learn python module\footnote{http://scikit-learn.org/stable/}, which will enable us to properly classify student learning data.

\end{document}