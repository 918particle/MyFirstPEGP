\documentclass[../../main.tex]{subfiles}
 
\begin{document}

\section{Background}

My name is Jordan Hanson, and I am formally submitting my first Professional Evaluation and Growth Plan (PEGP).  As required by Whittier College, and in accordance with the regulations in the Whittier College Faculty handbook, the material herein pertains to my first complete academic year as a tenure-track Assistant Professor of Physics and Astronomy.  Being new to the Whittier College community, I have included this professional introduction for those readers to whom I have not yet been introduced.  I look forward to meeting and working with my colleagues in other departments over the years, and I hope that this brief introduction will explain why I chose to become a professor.  Accordingly, I share my vision for teaching physics and scholarship in the area of \textit{astroparticle physics} at Whittier College. \\ \hspace{0.1cm}

My professors and colleagues in the professional-track physics program at Yale University inspired me to excel beyond what I thought was possible for myself.  I was introduced to the world of academic scholarship by faculty who had known they would enter this world from from a young age.  I fell in love with physics for the beauty of its theoretical simplicity, and the surge of excitement as observations spark to life through hard laboratory work.  After receiving my Bachelor of Science degree, I landed at UC Irvine, the home of the Nobel Laureate who made the first observation of a sub-atomic particle called a neutrino.  UC Irvine excels in the study of extrasolar, high-energy sub-atomic particles: \textit{astroparticle physics.}  I was introduced to Professors Steve Barwick and Stuart Kleinfelder.  Dr. Barwick is a professor of physics in the Department of Physics and Astronomy, and Dr. Kleinfelder is a professor of physics in the Department of Electrical Engineering.  Together we embarked on a journey to produce world record-breaking observations of high-energy neutrinos from beyond the solar system. \\ \hspace{0.1cm}

UC Irvine served as a training ground for my ability to teach, and I began to understand why teachers love to witness the flash of light in a student's eyes.  I taught as an assistant under Dr. Barwick, serving students in sections associated with introductory physics courses comprised of several hundred students.  During the early semesters in my graduate career, I was teaching physics sections of twenty students each for five continuous hours, three days per week.  After concluding my teaching duties, I focused on research for several years.  Upon completing my dissertation and receiving my doctorate, I solo-taught an introductory physics course during one of my post-doctoral fellowships.  During that summer I learned the difference between \textit{teaching} a course and \textit{creating} a course.  I enjoy creating new courses, and I have already created and taught new courses for students at Whittier.  Above all else, I hope my work at Whittier will serve to \textit{enlighten} our students.

\section{General Reflection and Future Directions}

Any general reflection for academics at Whittier must begin and end with our students.  Over the past year and a half, I have chosen to become an \textit{active participant in this community} and to push beyond what is required of me as a young professor.  I have taught introductory physics courses to students who have no prior experience in physics, and created a new advanced computer science course.  I've attended conferences to improve and expand my teaching methods, taking advantage of the broad research in physics education.  I decided to take a class from a professor in another department for the shear joy of learning, but also to learn methods from an experienced teacher.  I've involved a group of students in all facets of my research, including software and algorithm development, firmware development, and digital storytelling.  Two of these students won Keck Fellowships and have engaged in summer physics research in my laboratory.  We are preparing to become part of a collaboration of researchers who plan to build a world-class astroparticle detector at the South Pole.  Additionally, I've become a mentor and advisor to a student organization, and helped serve the Math Department in a tenure-track faculty search.  Each action I've taken during these past months has been geared towards serving our students thoughtfully and rigorously, to provide them with a quality education and research environment. \\ \hspace{0.1cm}

Despite these accomplishments, I am not satisfied with some aspects of my teaching.  I was surprised to find that in my introductory courses, a group of students felt that the level of mathematical and technical detail was too advanced, and that the pace of the courses was too rapid.  A group of students has been vocal in their assessment of these issues, and I take them seriously.  Some of my students in an introductory course actively worked with me in office hours to find common ground.  It is my hope that in the coming years, I will be able to implement a pace and difficulty level suitable for the academic environment at Whittier that \textit{does the most good, for the largest number of students}.  Although I do not feel it would be right to omit core physics principles from introductory courses, I will rely on the past experiences of my department to find a solution.  My hypothesis is that many of my students in introductory courses are not prepared to make logical abstractions of physical systems, and require a larger number of concrete examples and demonstrations before gaining that ability.  I will work diligently during the coming academic year to boost the abstract problem-solving ability of my students through leading by example.

\end{document}
