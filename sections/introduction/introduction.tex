\documentclass[../../main.tex]{subfiles}
 
\begin{document}

\section{Background}

My name is Jordan Hanson, and I am formally submitting my first Professional Evaluation and Growth Plan (PEGP).  As required by Whittier College, and in accordance with the regulations in the Whittier College Faculty handbook, the material herein pertains to my first complete academic year as a tenure-track Assistant Professor of Physics and Astronomy.  I have included this professional introduction for those readers to whom I have not yet been introduced.  I look forward to meeting and working with my colleagues in other departments over the years, and I hope that this brief introduction will both explain why I chose to become a professor, and share my vision for teaching and scholarship in the area of \textit{astroparticle physics} at Whittier College. \\ \hspace{0.1cm}

My professors and colleagues at Yale University inspired me to excel.  I was introduced to the world of academic scholarship by outstanding faculty, and I fell in love with physics.  The beauty of physics is found in its theoretical simplicity, and the surge of excitement as observations spark to life.  After receiving my Bachelor's of Science in physics, I became a graduate student at UC Irvine, the home of the Nobel Laureate who made the first observation of a sub-atomic particle called a neutrino.  UC Irvine excels in the study of extrasolar, high-energy sub-atomic particles: \textit{astroparticle physics.}  I was introduced to UCI professors Steve Barwick and Stuart Kleinfelder.  Dr. Barwick is a professor of physics, and Dr. Kleinfelder is a professor of electrical engineering.  We embarked on a journey to produce world record-breaking observations of high-energy neutrinos from beyond the solar system. \\ \hspace{0.1cm}

UC Irvine was a training ground for my teaching, and there I first witnessed the flash of understanding in students' eyes.  I taught as an assistant to Dr. Barwick, serving students in sections for introductory physics.  During the early semesters in graduate school, I taught sections of twenty students each for five continuous hours, three days per week.  After concluding my teaching duties, I focused on research.  Upon receiving my doctorate, I created an introductory physics course during a post-doctoral fellowship, learning the difference between \textit{teaching} a course and \textit{creating} a course.  I enjoy creating new courses, and I have already created and taught new courses for students at Whittier.

\section{General Reflection and Future Directions}

Reflection at Whittier must begin and end with the students.  Over the past year and a half, I have become an active participant in this community, pushing beyond what is required of me as a young professor in order to serve our students as best I can.  I have taught introductory physics courses to students who have no prior experience in physics, and created a new advanced computer science course.  I've attended conferences to improve and expand my teaching methods, taking advantage of the broad research in physics education.  I took a course from a professor in another department for the shear joy of learning, but also to learn methods from an experienced professor.  I've involved students in all facets of my research, and two of these students won Keck Fellowships while working in my laboratory.  We are preparing to become part of a collaboration of researchers building a world-class astroparticle detector at the South Pole.  Additionally, I won an award for mentorship to a student organization.  Each action I've taken during these past months has been geared towards serving our students thoughtfully and rigorously, to provide them with a quality education and research environment. \\ \hspace{0.1cm}

Despite these accomplishments, I am not yet satisfied with some aspects of my teaching.  I was surprised to find that in my introductory courses, some students felt that the level of technical detail was presented too rapidly and too advanced.  Some of my students in an introductory course actively helped me with these issues.  In the coming years, I will strive to implement a pace and difficulty level suitable for the academic environment at Whittier that \textit{does the most good, for the largest number of students}.  Although it would not be right to omit core physics principles from introductory courses, I will rely on the past experiences of my department and listen to the students.  My hypothesis is that some students in introductory courses are unaccustomed to making logical abstractions required in introductory physics, requiring a slower pace and concrete examples before gaining that ability.  I will work diligently to boost the abstract problem-solving ability of my students through leading by example.

\end{document}
