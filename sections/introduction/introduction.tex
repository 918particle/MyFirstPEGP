\documentclass[../../main.tex]{subfiles}
 
\begin{document}

\section{Background}

My name is Jordan C Hanson, and I am formally submitting my first Professional Evaluation and Growth Plan (PEGP).  As required by Whittier College, and in accordance with the regulations in the Whittier College Faculty handbook, the material herein pertains to my first complete academic year as a tenure-track Assistant Professor of Physics and Astronomy.  Being new to the Whittier College community, I have included professional background material to serve as an introduction for those readers to whom I have not yet been introduced.  I look forward to meeting and working with my colleagues in other departments over the years, and I hope that this brief introduction will explain why I chose to become a professor.  Accordingly, I share my vision for teaching physics and scholarship in the area of \textit{astroparticle physics} at Whittier College. \\ \hspace{0.1cm}

My professors and colleagues in the professional-track physics program at Yale University inspired me to excel beyond what I thought was possible for myself.  I was introduced to the world of academic scholarship by faculty who had known they would enter this world from from a young age.  I fell in love with physics for the beauty of its theoretical simplicity, and the surge of excitement as observations spark to life through hard laboratory work.  After receiving my Bachelor of Science degree, I landed at UC Irvine, the home of the Nobel Laureate who made the first observation of a sub-atomic particle called a neutrino.  At UC Irvine, a group of professors focus on the study of extrasolar, high-energy sub-atomic particles: \textit{astroparticle physics.}  I was introduced to Professors Steve Barwick and Stuart Kleinfelder.  Dr. Barwick is a professor of physics in the Department of Physics and Astronomy, and Dr. Kleinfelder is a professor of physics in the Department of Electrical Engineering.  Together we embarked on a journey to produce world record-breaking observations of high-energy neutrinos from beyond the solar system. \\ \hspace{0.1cm}

UC Irvine served as a training ground for my ability to teach, and I began to understand why teachers love to witness the flash of light in a student's eyes.  I taught as an assistant under Dr. Barwick, serving students in sections associated with introductory physics courses comprised of several hundred students.  At one point during an early semester in my graduate career, I was teaching physics sections of twenty students each for five continuous hours.  I learned to adapt to students of various strengths, and to keep my material organized and class on-task.  After concluding my teaching duties, I focused on my research for several years.  Upon completing my dissertation and receiving my doctorate, I solo-taught an introductory physics course during one of my post-doctoral fellowships.  This experience was vital because it showed me the difference between \textit{teaching} a course and \textit{creating} a course.  I enjoy creating courses, and I have already created and taught new courses for students at Whittier.

\section{General Reflection and Future Directions}

Any general professional reflection for academics at Whittier must begin and end with our students.  

\end{document}
