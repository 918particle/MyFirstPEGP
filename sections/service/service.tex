\documentclass[../main.tex]{subfiles}
 
\begin{document}

When professors usually discuss service to Whittier College, it revolves around work done on comittees.  Service on committees is the manifestation of the idea that a university should be self-governed by the faculty.  Partnering with the administration, Whittier College faculty have a responsibility to shepherd the institution forward into the future.  I take this responsibility seriously, and although I was not assigned a commitee during my first year as an assistant professor, I have volunteered to serve Whittier College in five ways. \\ \hspace{0.1cm}

First, I served as a faculty advisor to a student organization.  In recognition of my service, the students gave me The Outstanding Organization Advisor Award for a student organization.  Second, I volunteered to serve on a Faculty Search Committee in the Department of Mathematics, which led to the selection of Dr. David Claveau.  Third, I've helped to both recruit new physics majors and to serve current physics majors by attending campus recruitment events and serving the Society of Physics Students.  Fourth, I continue my usual program of giving public lectures, which is meant to serve Whittier College by increasing our recognition in the surrounding Los Angeles community.  Finally, I've joined the Enrollment and Student Affairs Committee (ESAC), with a focus on analyzing admissions data.  Below I outline a project I've recently undertaken in service to ESAC. \\ \hspace{0.1cm}

\section{Service to Student Organizations - CRU}

As a convert to the Christian faith, and a member of the Catholic Church, I believe that my purpose in life is to glorify and love the Lord by obeying The Golden Rule - ``Love your neighbor as yourself,'' from The Gospel of Matthew (7:12 and 22:36-40).  Whitter College has a student organization called Campus Crusade, which is colloquially known as CRU.  CRU is actually a national organization that provides Christian ministry to college students.  The Whittier chapter of CRU usually has 10-20 members each year, and meets weekly to read the Bible, sing, and have fellowship by building a community.  Each year, CRU also creates Bible study groups, traditionally partitioned into the Men's Bible Study and the Women's Bible Study. \\ \hspace{0.1cm}

I can relate to the experience of CRU members.  I have kept my faith through the adventure of graduating from college, graduate school, and finally becoming a professor.  In addition to the necessary examination of the Christian faith that naturally occurs in a liberal arts college, the college experience in America can at times pull young people away from theology, philosophy, and religious practice.  I experienced professors who derided Christianity when I was their student in college.  Rather than debating students, I simply accept their invitation to read and analyze scripture in fellowship alongside them.  During the weekly Men's Bible Study last year, I joined typically 5-7 students to read the letter of St. Paul to the Romans, in the New Testament. \\ \hspace{0.1cm}

I can confidently say that we all grew in our spiritual knowledge after finishing Romans, and that I am especially proud of those young men from my Bible study group who graduated in Spring 2018 with the moral support of the group.  Some of these students were my physics students, and their friends fellow CRU members continue to take my classes.  What I find most important is to empathize with the students, and find ways to serve them in their growth as spiritual people alongside their intellectual growth.  \\ \hspace{0.1cm}

At the end of the Spring 2018 semester, the student organizations held an award ceremony to recognize students and faculty who have gone above and beyond normal service to Whittier College and the surrounding community.  The student government recognized me with The Outstanding Organization Advisor Award, for my service to CRU.  It was a proud moment in my first year at Whittier College.  I am the first professor to win this award in its inaugural year.  I will continue to encourage my colleagues to connect with the students in new and inspiring ways, and I look forward to the award ceremony next spring. \\ \hspace{0.1cm}

\section{Faculty Search for the Department of Mathematics}

I was brought on to the Department of Physics and Astronomy at Whittier College to help add value to our computer science curriculum, in addition to teaching physics courses.  The computer science curriculum is currently delivered to the students by a combination of physics and mathematics professors.  Since the Department of Mathematics identifies me as a stakeholder in the future computer science curriculum, I was asked to serve on the committee to select a new tenure-track mathematics professor.  This new professor is expected to teach not only mathematics, but robotics and other forms of computer science.  I participated in many activities as part of this tenure-track search.  \\ \hspace{0.1cm} 

I attended faculty search training led by the Dean of the Faculty, Darrin Good.  I attended Mathematics Department meetings, and also I attended planning meetings to draft the job listing and interview questions, I helped schedule phone calls with the applicants, and I conducted phone interviews with finalist candidates.  Once final candidates were invited to campus, I interviewed them one-on-one in my office.  Along with Professor Fritz Smith, I interviewed candidate David Claveau over the phone.  I was happy when David was nominated for the position.  I look forward to collaborating with David in the field of robotics and drones, which have come up recently in my research. \\ \hspace{0.1cm}

\section{Recruitment and Service to the Physics Department}

I have directly served my department in two major fashions.  The first involves recruitment and mentoring.  Along with Professor Seamus Lagan, I have attended recruitment events and mentoring events for physics majors admitted to Whittier College.  The first such event was held in the student center in Spring 2018.  Professor Lagan and I ran a booth distributing information to admitted incoming freshmen about majoring in 3-2 engineering, and physics.  We seek to recruit new physics majors at events like these, and once they arrive on campus, we engage in the mentoring program. I also helped current physics majors run the annual Star Party, when we took 30 Whittier undergraduates to the desert of Joshua Tree to observe stars and galaxies.  \\ \hspace{0.1cm}

The second form of service directly tied to physics is my volunteer work with the Artemis program.  Professor Serkan Zorba pitched the idea to me, and I have volunteered for the current academic year as an Artemis program sponsor.  Physics is infamous for gender disparity, and programs like Artemis seek to balance the disparity by demonstrating to high-school aged female students that careers in physics are for them as well as the male students.  I will be connecting my Primer project with Artemis, as I think it nicely dovetails conceptually.  It is always a good idea to expose high school students to programming, as many do not have access to programming classes yet in high school but do need that skill in college.  I have already received approval of my abstract from Sam Ruiz\footnote{For more information, contact Sam Ruiz: sruiz3@whittier.edu}, my Artemis coordinator and liason. \\ \hspace{0.1cm}

\section{Public Lectures}

Professor Seamus Lagan also connected me with a local high school student, Cole Aedo, who has created STEM Lecture Nights.  I have already given a public lecture as part of this program at a local middle school, Los Nietos Middle School in Los Nietos, CA.  I created an hour-long lecture that provided content on my field of astroparticle physics tailored to the audience of middle school STEM students and their parents.  I also answered questions and promoted Whittier College as a place that will provide them with both a good education and an opportunity to do quality scientific research.  Giving public lectures is something I've always done, and my C.V. gives other examples of lectures in the past.

\section{Enrollment and Student Affairs Committee}

Finally, I have joined the Enrollment and Student Affairs Committee (ESAC).  I have taken minutes and attended meetings, and therein was assigned to a sub-committee examining admissions data.  Given that I have a background in analytics and know how to wield machine-learning algorithms, I've embarked on an analytics project seeking to understand recent admissions data and provide insights to the Admissions office.  With me on this project are Professors Chuck Hill and Fritz Smith.  We are currently collecting the data, which will focus on the last five years of admissions.  \\ \hspace{0.1cm}

We seek to derive insights into student outcomes, with the final goal of predicting things like the likelihood students will matriculate to their sophmore year, or their growth (quantified by measureble parameters).  To achieve this, I proposed to use genetic algorithms to derive a function from the data that takes as inputs data classifiers such as SAT scores, freshman year GPA, and academic quintile classification (AQ1-5).  Genetic algorithms have the ability to try vast combinations of linear \textit{and non-linear} combinations of data classifiers until the best combination is found that correctly predicts the outcome.  If successful, ESAC would be able to provide the Admissions Office with a protocol that will predict student success.  An example would be an algorithm that predicts the likelihood Whittier admittees will matriculate to sophmore year.  In consultation with members of the Inclusion and Diversity Committee (IDC), I have begun to think about ways in which we might locate and control for factors that carry implicit bias.

\end{document}
