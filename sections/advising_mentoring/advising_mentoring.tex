\documentclass[../main.tex]{subfiles}
 
\begin{document}

Serving the Whittier students in the role of advisor is one that I take seriously, especially in light of the fact that majoring in physics is a difficult path to choose.  Often students who excel at computer programming, mathematics, and science in general do not realize how good of a fit they would be in the physics program because the tangible benefits of majoring in physics are not always obvious.  The truth is that majoring in physics opens many technical career paths.  Thus, the first task in advising students in physics is to convince them that physics is a good option.  The second facet of physics advising is to help physics students to discern the types of physics they should take at the advanced level.  Physics courses fall into the \textit{experimental} and \textit{theoretical} groups.  Although we require students to take a minimum number of courses from each category (as well as a basic training in advanced mathematics), the advanced technical courses a student will need depends largely on their future research or professional plans.  The final facet of advising is to guide the student through a senior research project (officially listed as Senior Seminar in Physics, PHYS499A/B) that exposes them to the year-long research process.  Below I make the case that I have participated in each phase of physics advising. \\ \hspace{0.1cm}

In Spring 2018 I participated in a weekend recruitment fair to provide newly admitted students information about majoring in physics and astronomy.  I also helped Professor Seamus Lagan recruit 3-2 engineering program majors, as there is significant overlap in course load and planning between the 3-2 program and physics/mathematics.  I spoke with many students and parents about the tangible benefits of choosing to major in physics, and choosing Whittier College.  I emphasized that physics opens many doors, even if the student chooses not to enter academia.  Professor Lagan also invited me to help him with the freshmen mentoring program during Fall 2018 orientation, in which we guided new physics students through topics like becoming accustomed to campus and course enrollment.  These meetings and ice-breaking activities have been fruitful; several students have now approached me to discuss my research and how they might participate. \\ \hspace{0.1cm}

I have helped to recruit at least two physics majors.  Cassady Smith has switched to physics and is now in her sophomore year.  It seemed that she was considering majoring in physics, and after taking my calculus-based physics course and writing an extra-credit essay for me on the Advanced LIGO project (the Nobel Prize for the discovery of gravity waves), she decided to double major in French and Physics.  I also convinced her to take my COSC330 (Computer Logic and Digital Circuit Design).  Cassady is not my formal advisee yet, but worked with me as a Keck Fellow over Summer 2018 and is still actively doing research with me.  Nicholas Clarizio has completed a business major and is attempting to double major.  Nicholas Clarizio is now my formal physics advisee, and I am guiding him through the second and third facets of physics advising: advanced course selection and choosnig a senior project.  Currently, I am having him do research for credit for me (listed as PHYS396) in anticipation of a larger project for a senior seminar.  Nicholas and I are building an RC drone together, for Antarctic operations.  My hope is that as he prepares for graduation next year, this research will blossum into a year-long project that will help him to graduate as my first official physics advisee. \\ \hspace{0.1cm}

With Cassady Smith and John-Paul G\'{o}mez-Reed, my two Keck Fellows, I do a great deal of \textit{informal advising}, that is, mentoring.  I attempt to mentor them in the area of professional growth by aiding them in project design and application to research programs beyond Whittier College.  This is also true to a lesser extent of a student named Nicholas Haarlammert, who took calculus-based physics from me with Cassady.  I wrote letters of recommendation to several physics research experiences for undergraduates (REUs) for Cassady and Nicholas Haarlammert, including for the University of Michigan and Caltech\footnote{These letters of recommendation are included in the supplemental material.}.  Cassady wound up doing the Keck Fellowship here at Whittier, as physics internships are competitive and usually go to juniors and seniors with more experience.  I see mentoring as an extension of advising, in that we must advise students about how to proceeed with their physics degree after college, in addition to getting them to graduation day.  I have introduced both Cassady Smith and John-Paul G\'{o}mez-Reed to my research colleagues at UC Irvine, in the hopes that they might look into going to graduate school there.  Both studets would make excellent PhD candidates.


\end{document}
