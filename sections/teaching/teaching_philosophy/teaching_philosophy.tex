\documentclass[../../main.tex]{subfiles}
 
\begin{document}

\epigraph{\textit{The heart of the intelligent acquires knowledge, and the ear of the wise seeks knowledge.} - Proverbs 18:15 \\ \vspace{0.1cm} \textit{I guess you could call it a ``failure,'' but I prefer the term ``learning experience.''} - Astronaut Mark Watney in \textit{The Martian} by Andy Weir}{}

\textbf{Teaching is about beautiful failure}.  Learning is an activity that takes place between at least two people where at least one lacks knowledge.  Learning is beautiful because a lack of knowledge is an advent to \textit{enlightenment}.  Regardless of the teaching methods chosen for a given teacher and student, the student should leave the encounter \textit{enlightened}, with increased knowledge of the truth.  The success of the encounter is measured by the varying degree to which the student can retain, apply, understand, and reflect upon the knowledge.  I believe that lifting a student learning physics from retention to reflection is beautiful, in that I witnesses a student extending their mind outside \textit{their model} of the world, into \textit{the model} of the world.  In general, both the teacher and student succeed imperfectly in imparting ideas about \textit{the model} of nature, and therefore the learning process will contain periodic failures.  Further, the physics model itself may be an imperfect description of true nature.  Acknowledging and growing past these ``failures'' is a hallmark of learning modern physics, a subject built upon increasingly accurate approximations to the truth. \\ \hspace{0.1cm}

Teaching physics begins with defining the concept of a ``system'' about which we can make measurements.  Physics majors and non-majors alike must all begin at the same place.  With well-defined concepts of distance, displacement, and time, the entire subject of \textit{classical physics} may be undertaken.  Students who are non-majors who do not take specialty courses usually experience exclusively classical physics.  Physics majors grown through the inaccuracies of classical physics to \textit{modern physics}, which includes relativity and quantum mechanics.  Mastering these subjects represents a scientific maturity only possible through diligent and patient teaching.  Teachers who are both capable of bringing students to the advanced level and enlightening beginner students are not molded upon the completion of graduate school.  Mastery of physics teaching requires experiences shaped by the failure and success of elevating a broad variety of students.  During the past year, I have gained valuable experience in teaching both types of students. \\ \hspace{0.1cm}

\textbf{A good teacher loves growth}.  Each semester at the beginning of my introductory courses, I give a speech about learning to embrace failure entitled ``It's OK to Be Wrong.''  The introductory student fears being wrong, losing points, and receiving a low grade.  Counter-intuitively, those students who embrace their mistakes and learn from them turn out to be the strongest students.  Converting failure to growth has two components.  First, there is no substitute for \textit{hard work and sacrifice}.  A good teacher leads by example, pouring effort into the semester until the job is done.  A good teacher works to master new skills by attending teaching conferences in his field, consulting students through mid-semester feedback mechanisms, analyzing student evaluations.  A good teacher also works to become nimble, switching from method to method, until the suitable vehicle properly engages the student.  Second, a good teacher \textit{creates a proper learning space}.  In my classrooms, no student is penalized for being wrong, with the single exception of taking exams.  By creating a space in which it is ok to be wrong, we take advantage of the learning moments brought forward by mistakes, and make real progress. \\ \hspace{0.1cm}

A good \textit{professor} is a special kind of teacher, in that he is a teacher that also performs scientific research.  A good professor successfully involves undergraduate students in his research.  One crucial fact about myself that I learned during the past two semesters is that I love the \textit{instructive} act of research just as much as I love the \textit{investigative} act.  Even when I am conducting research with my students, I should still be instructing them, and I've found that I love it.  The instructive act of research lies in \textit{pausing to reflect} upon what our actions in the laboratory imply.  Whether a procedure succeeds or fails my laboratory, the student and I must take time away from the procedure to step back and understand \textit{why} we observed the result.  I hope to grow as much in the area of research instruction as I will grow in classroom instruction, and to produce students who will become quality researchers.

\subsubsection{Instruction of Students in Introductory Courses}

The first categorization of physics student at Whittier College is whether they are a liberal-arts \textit{non-major} or \textit{physics major}.  Non-majors encounter physics for two semesters in either a \textit{calculus-based} or \textit{algebra-based} environment.  We categorize students in this fashion because classical physics at the standard undergraduate introductory level is built upon single-variable calculus, with some multi-variable or vector calculus introduced in the second semester.  Students who will not take calculus for their degree can still learn core mathematical concepts like vectors and instantaneous quantities and apply them to physics.  Thus, \textit{non-major} students usually take the \textit{algebra-based} version of mechanics, and \textit{physics majors} and students who have chosen another technical degree take the \textit{calculus based} version of mechanics.  \\ \hspace{0.1cm}

It is important for three reasons to deliver a quality physics experience to non-majors in algebra-based physics:
\begin{enumerate}
\item A
\item B
\item C
\end{enumerate}

\subsubsection{Instruction of Students in Upper-division Courses}

\end{document}

