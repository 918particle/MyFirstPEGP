\documentclass[../../main.tex]{subfiles}
 
\begin{document}

\epigraph{\textit{The heart of the intelligent acquires knowledge, and the ear of the wise seeks knowledge.} - Proverbs 18:15 \\ \vspace{0.1cm} \textit{I guess you could call it a ``failure,'' but I prefer the term ``learning experience.''} - Astronaut Mark Watney in \textit{The Martian} by Andy Weir}{}

\textbf{Teaching is about growth through failure}.  Learning takes place between at least two people where at least one seeks knowledge.  Seeking knowledge is an advent to \textit{enlightenment} and is therefore beautiful.  Regardless of the teaching methods chosen for a given teacher and student, the student should leave the encounter \textit{enlightened}, with increased knowledge of the truth.  The success of the encounter is measured by the varying degree to which the student can retain, apply, understand, and reflect upon the knowledge.  Lifting a student learning physics from retention to reflection is beautiful, in that we witness students extending their minds outside \textit{their model} of nature, into \textit{the model} of nature.  In general, the teacher and student succeed imperfectly in imparting ideas about \textit{the model} of nature.  Thus the process will contain periodic failures.  Growing through these ``failures'' is a hallmark of learning modern physics, a subject built upon increasingly accurate approximations to the truth. \\ \hspace{0.1cm}

Teaching physics begins with defining the concept of a ``system'' about which we can make measurements.  All physics students must begin at this common place.  With well-defined concepts of distance, mass, displacement, and time, the entire subject of \textit{classical physics} may be undertaken.  Students who are non-majors usually experience classical physics.  Physics majors grow through the inaccuracies of classical physics to \textit{modern physics}, which includes relativity and quantum mechanics \footnote{Students satisfying liberal arts requirements via specialty courses do experience non-classical physics qualitatively.}.  Mastering these subjects represents a maturity made possible through diligent and patient teaching.  Teachers capable of bringing students to the advanced level and enlightening beginners are not molded upon the completion of graduate school.  Physics teaching requires experiences shaped by failures and successes enlightening students studying classical and modern physics. \\ \hspace{0.1cm}

\textbf{A good teacher loves growth}.  Each semester at the beginning of my introductory courses, I give a speech about learning to embrace failure entitled ``It's OK to Be Wrong.''  The introductory student fears being wrong, losing points, and receiving a low grade.  Counter-intuitively, those students who embrace their mistakes and learn from them turn out to be the strongest students.  Converting failure to growth has two components.  First, there is no substitute for \textit{hard work and sacrifice}.  A good teacher pours work into the semester, and masters new skills in the field.  A good teacher also works to become nimble, switching from method to method, until the suitable vehicle properly engages the student.  Second, a good teacher \textit{creates a proper learning space}.  In my classrooms, no student is penalized for being wrong, with the single exception of taking exams.  By creating a space in which it is ok to be wrong, we make progress at the learning moments brought forward by mistakes. \\ \hspace{0.1cm}

A good \textit{professor} is a special kind of teacher, in that he is a teacher that also performs scientific research and serves a college or university.  A good professor successfully involves students in his research.  One crucial fact I learned during the past two semesters is that I love the \textit{instructive} act of research just as much as I love the \textit{investigative} act.  While conducting research with my students, I should still be instructing them, and I've found that I love it.  The instructive act of research lies in \textit{pausing to reflect} upon what our actions in the laboratory imply.  Whether a procedure succeeds or fails my laboratory, the student and I take time to understand \textit{why} we observed the result.  I hope to grow as much in the area of research instruction as I will grow in classroom instruction, and to produce students who will become quality researchers.

\subsubsection{Instruction of Students in Introductory Courses}

\label{sec:teaching_phil1}

Physics students at Whittier College are first categorized as liberal-arts \textit{non-majors} or \textit{physics majors}.  Non-majors encounter physics for two semesters in either \textit{calculus-based} or \textit{algebra-based} courses.  We provide such introductory courses because classical physics at the undergraduate introductory level is built upon single-variable calculus, with some multi-variable or vector calculus introduced in the second semester.  However, students who will not take calculus for their degree can still learn to apply some mathematical concepts.  Thus, \textit{non-major} students usually take the \textit{algebra-based} version of mechanics, and \textit{physics majors} and students who have chosen another technical degree take the \textit{calculus based} version of mechanics.  \\ \hspace{0.1cm}

Three focuses are relevant for teaching non-majors algebra-based physics:
\begin{enumerate}
\item \textbf{Curiosity}.  I regularly give colloquia at universities, seminars in physics departments, and public lectures to children, families, and astronomical societies.  Experiencing people's curiosity is necessary to become a great professor.  I've continued this practice as Whittier professor, for example, by giving a lecture at Los Nietos Middle School.  All people seek an understanding of nature.  Moreover, people have a need to know \textit{that the answers exist}, even if we do not all fully grasp them.  I believe good teaching for non-majors should therefore \textit{convince them that physics is interesting} by enticing their curiosity.  I have built into the algebra-based curriculum specific learning activities designed to entice student curiosity.  Presenting science articles to the class and presentations on home-built circuit projects are two examples.  I regularly give colloquia at Whittier and incentivize my students to participate, exposing them to astroparticle physics research \footnote{See supporting materials for notes from students on my spring colloquium.}.

\item \textbf{Improvement of Analysis Skill}.  The scientific method relies on analysis.  We as physicists best serve Whittier non-majors when we are developing their problem-solving.  The Whittier College physics faculty have several tools for problem-solving development.  \textit{Peer Instruction} is becoming a standard method in American colleges \cite{mazur}.  \textit{Just in Time Teaching} is an auxiliary method designed to modify class time, focusing on the problem solving strategies the students find challenging \cite{howpeoplelearn}.  A third tool is PhET (Physics Education Technology) \cite{phet}, in which students compare analysis results to computer simulations.  We employ an integrated lecture/laboratory format, which is facilitated by the design of the Science and Learning Center.  The integrated techniques allow the instructor to provide diverse activities, such as group problem solving, verification with computer simulations, and verification via experimentation.  Finally, we incorporate a healthy mix of \textit{traditional} lecture methods to provide concrete examples for our students encountering content for the first time \footnote{Traditional lecture methods refer to a broad class of instruction methods, but generally refer to the professors performing example calculations on the chalkboard while students take notes and learn through repetition.}.

\item \textbf{Applications to Society}. Whittier College non-majors gain potential in technically oriented careers if they can qualitatively explain phenomenon using physics.  In recent years, our standard open-source textbooks have included material relevant to popular majors (e.g medicine and KNS). I have incorporated special units centered on these applications, including human nerve systems (in PHYS135B) and human metabolism (PHYS180).  I use the final group projects to allow non-majors to present on the intersection of physics and their field of study.  I proposed a new course entitled \textit{Physics of the Five Senses}, designed to be connected with KNS courses.  I plan to reintroduce this course in the near future when appropriate \footnote{The KNS department did not have the personel for pairing or team-teaching.  Subsequently, the number of new students requiring PHYS180 increased, and we had to drop my CON2 plans so I could add a section of PHYS180.}. I included also a brief unit on climate change and the solar system in PHYS135B and PHYS180, based on analysis and simulations.  One additional tool for the non-majors is the inclusion of student-led summaries of a scientific journal article in 5-10 minutes.  The brevity requirement causes the students to focus on important details, and the societal implications.
\end{enumerate}

\subsubsection{Instruction of Students in Advanced Courses}

\label{sec:teaching_phil2}

\textit{Physics majors} are the second category of students we typically encounter.  I broaden my discussion to \textit{Mathematics and Computer Science majors} due to the specific circumstances under which I was hired.  The Departments of Mathematics and Physics at Whittier College provide the current computer science curriculum.  Our students can choose majors that combine computer science with physics, math, or economics, or join the 3-2 engineering program.  The upper-level computer science course I created is Computer Logic and Digital Circuit Design (COSC330/PHYS306).  Those who participated were physics majors, mathematics majors, and Whittier Scholars Program majors, all having some connection to computer science.  This course is under rapid development in parallel with developments in my research laboratory (see section on Scholarship below). \\ \hspace{0.1cm}

Three focuses are relevant for teaching physics, mathematics, and computer science majors at the advanced level, in addition to those above for non-majors and introductory courses:
\begin{enumerate}
\item \textbf{Mental Discipline}.  Advanced physics, math and computer science courses require discipline. and there is no substitute for grit.  I think the professor has a role in calling forth this value in the students, in two ways.  The first is delivering a rigorous curriculum.  Problem sets and exams should be difficult, requiring time and reflection.  For example, in COSC330, homeworks were assigned in two-week increments, with both mathematical repetition (to facilitate learning binary) and open-ended design questions (like designing a device that sums binary numbers).  Second, the content delivery should demonstrate expertise, but also show the students that the professor is invested in them.  Advanced classes in large universities sometimes leave the students with a blunt delivery that merely entices them to teach themselves.  The right path leaves the student \textit{motivated} to fill in gaps in their understanding, with the professor thoughtfully elevating students' understanding outside of class.  For example, in COSC330 my students and I happily debugged digital circuits in simulation software together, before building them for class presentation.

\item \textbf{Strength in all Phases of Science}. Good curriculum in these advanced topics must include the following \textit{phases} of scientific activity: theoretical problem solving, numerical modeling or simulation, experimental design and execution, and data analysis.  We may think of these phases as the actions that move the student through the scientific method.  In COSC330, an example of the incorporation of all four phases occurs in teaching the students to work with binary numbers and code.  First, the mathematics for conversion from decimal to binary is introduced along with addition and subtraction techniques, and we work example problems.  Second, we model addition and subtraction via 8-bit adders in a computer simulation.  Third, we actually build the adders, and fourth, we demonstrate that they work by analyzing the outputs.  When students gain experience in all four phases, they more firmly grasp the concept.  Students are also more likely to have a breakthrough in understanding a concept if they encounter it in multiple phases.

\item \textbf{Communication}.  Two skills that should never go overlooked in technical fields are oral and written communication.  Presentations, papers, lab reports, and summarizing peer-reviewed articles for the class are several examples of rubrics that I use in advanced courses to hone communication skills.  From personal experience, work in technical subjects would often proceed more quickly if not for the inability of group members to express themselves clearly.  When dealing with abstract concepts in engineering discussions, clear communication prevents the introduction of design flaws and the introduction of bugs in software. No matter which advanced class I am teaching, my students will write at least one report, or give one presentation.  I often allow students to write for extra credit, going beyond the scope of the course in the subject matter.  Any practice in technical writing Whittier majors receive now will benefit them down the line as they proceed to graduate school or private sector engineering careers. \footnote{See supplemental materials for examples of student presentations and writing.}
\end{enumerate}

\subsubsection{Department-Level Goals}

The Department of Physics and Astronomy has eight goals, developed as part of our 5-year assessment cycle. In the coming course descriptions, these goals will be referenced. The departmental goals are:

\begin{enumerate}
\item Develop and offer a wide range of physics courses using the most effective pedagogical methods and styles.  Such courses shall include appropriate contributions to the Liberal Education Program (currently COM1 and CON2).
\item Create research experiences for physics majors that will engage and inspire them in their discovery of physics.
\item Build a departmental community that is supportive and welcoming and that encourages students in their studies of physics.
\item Keep the physics curriculum current so that students gain the skills necessary for success in today’s scientific environment.
\item Teach students how to teach themselves. Give them the intellectual tools necessary for independent thinking and learning.
\item Train students to think ``scientifically'' i.e. critically, rigorously, quantitatively, and objectively, so that they can analyze problems and generate solutions.
\item Train students to effectively communicate scientific ideas to others.
\item Advise students about various career paths and help them along these paths.
\end{enumerate}

\end{document}

