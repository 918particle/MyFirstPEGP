\documentclass[../../../main.tex]{subfiles}
 
\begin{document}

\textbf{\textit{Computer Logic and Digital Circuit Design}}. My premier advanced course was ambitious, and has a well-defined direction for continual improvement.  Digital design is as broad a topic as any undergraduate would encounter.  To cover it adequately at Whittier College, I had to make hard choices about where to spend class-time. My first goal for the students was to impart my advanced learning focus of \textbf{strength in all phases of science}, and to satisfy departmental goals 4-7. Naturally multi-disciplinary, digital design has many sub-topics\footnote{See supplemental material for a course syllabus.  Although listed as COSC330, this course is also cross-listed as PHYS306, so I felt our departmental goals should apply.}.  COSC330/PHYS306 is a 300-level integrated computer science course that satisfies core requirements in the following majors: ICS/Math, ICS/Physics, ICS/Economics, 3-2 Engineering/Math, and the scientific computing minor.  Such a broad course that serves a wide variety of students should touch on at least the following sub-topics:

\begin{enumerate}
\item Binary mathematics and non-decimal base systems
\item Boolean algebra and logic
\item Implementing boolean algebra with transistors
\item Digital clock signals and digital component specifications
\item Digital components built from clocks and transistors
\item Complex digital systems (microprocessors, microcontrollers)
\end{enumerate}

Additionally, any good digital design course at a liberal arts college must evenly cover the following phases of the field: \textit{mathematics, computer programming, hardware design and function, and computer modeling.}  I attempted to design a course syllabus that incorporated \textbf{all phases} of the field. \\ \hspace{0.1cm}

My first advanced course learning focus is \textbf{mental discipline}, and I attempted to reach that goal in three ways.  First, the homework assignments were difficult, and assigned in two-week increments, with both mathematical repetition (to facilitate learning to speak with binary and boolean algebra) and open-ended design questions \footnote{See supplemental material for examples of assignments.}.  Second, I chose to combine a traditional lecture component with electronic slides that meshed with my work on the whiteboard, as I've observed with professors of foreign language at Whittier College.  Teaching binary to newcomers felt like teaching a new programming or spoken language.  Solving problems individually and in pairs helped those struggling during the introduction of Boolean algebra.  A language course requires a student \textit{verbally communicate} repeatedly with others to improve grammar and comprehension, so I required my students to practice the same mental discipline.  Finally, I assigned design problems in the homework and group projects.  The students' project designs had to achieve an agreed-upon task, via a project proposal.  Next, they had to be modeled with software, built, tested, and presented to the group.  The group projects were designed also as additional oral and written communication practice. \\ \hspace{0.1cm}

My second advanced course learning focus is \textbf{strength in all phases of science}.  By design, this course incorporated multiple sub-topics and four phases (as listed above): \textit{mathematics, computer programming, hardware design and function, and computer modeling.}  The first month of the course required me to focus on binary math and boolean algebra, finishing with the topic of Karnaugh maps \footnote{Karnaugh maps are a way of speeding through boolean algebraic derivations efficiently.}.  Unfortunately, although I ordered the digital components for this course over a month in advance, the purchase orders were not followed and we did not received the parts until halfway through the course.  This disrupted my curriculum, but the parts are reusable so it cannot happen again.  While waiting for components, we focused on simulating the circuits implied by our algebraic derivations with a software package called LogicWorks.  LogicWorks gave the students the benefit of seeing how their designs would behave over time, once activated.  It also allowed student to locate rare cases for which the algorithm implied by their design would fail.  When the hardware arrived, we built everything from super-heterodyne AM transistor radios, to circuits that could add two 8-bit binary numbers.  The students enjoyed the tinkering aspect of the course in the lab, but I would have wanted the lab and lecture activities to be more integrated. \\ \hspace{0.1cm}

My third advanced learning focus is \textbf{communication}.  I had groups of two and three submit project proposals to me for approval.  The project proposal rubric is graded on \textit{attention to detail}, and the students responded with diagrams, sketches, and text explaining their design.  Once everyone was on the same page, I required them to simulate the design in LogicWorks, checking it for flaws, and to show me progress.  The students used office hours to ask for help ``de-bugging'' their designs, which is jargon for trouble-shooting.  I recall spending a few hours with each group during the semester thinking about their design logic in an attempt to locate algorithm flaws.  In these moments I admired the growth in the students' mathematical communication.  The final presentations were good, and would have been excellent had I provided more specific guide-rails for the presentation content.  What we experienced was two main presentation components: the demonstration and the explanation with data.  In the future, I will formalize the requirement of both, and provide a structured schedule for the components of the assignment. \\ \hspace{0.1cm}

\end{document}

