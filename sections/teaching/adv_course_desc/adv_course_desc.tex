\documentclass[../../../main.tex]{subfiles}
 
\begin{document}

\textbf{\textit{Computer Logic and Digital Circuit Design}}. My premier advanced course was ambitious, and has a well-defined direction for continual improvement.  Digital design is as broad a topic as any undergraduate would encounter.  To cover it adequately at Whittier College, I had to make hard choices about where to spend class-time. My first goal for the students was to impart my advanced learning focus of \textbf{strength in all phases of science}, and to satisfy departmental goals 4-7. Naturally multi-disciplinary, digital design has many sub-topics within it\footnote{See supplemental material for a course syllabus.  Although listed as COSC330, this course is also cross-listed as PHYS306, so I felt our departmental goals should apply.}.  COSC330/PHYS306 is a 300-level integrated computer science course that satisfies core requirements in the following majors: ICS/Math, ICS/Physics, ICS/Economics, 3-2 Engineering/Math, and the scientific computing minor.  Such a broad course that serves a wide variety of students should touch on at least the following sub-topics:

\begin{enumerate}
\item Binary mathematics and non-decimal base systems
\item Boolean algebra and logic
\item Implementing boolean algebra with transistors
\item Digital clock signals and digital component specifications
\item Digital components built from clocks and transistors
\item Complex digital systems (microprocessors, microcontrollers)
\end{enumerate}

Additionally, any good digital design course at a liberal arts college must evenly cover the following phases of the field: \textit{mathematics, computer programming, hardware design and function, and computer modelling.}  I attempted to design a course syllabus that incorporated \textbf{all phases} of the field.

\begin{table}
\centering
\begin{tabular}{| c | c | c | c | c |}
\hline \hline
Semester & Course & Credits & Students & Curriculum feature \\ \hline
Fall 2017 & PHYS135A-01 & 4.0 & 24 & None \\ \hline
Fall 2017 & PHYS150-01 & 4.0 & 17 & COM1 \\ \hline
Spring 2018 & PHYS135B-01 & 4.0 & 18 & None \\ \hline
Spring 2018 & PHYS180-02 & 5.0 & 19 & COM1 \\ \hline
Spring 2018 & COSC330/PHYS306 & 3.0 & 6 & Advanced course \\ \hline
-- & Total & 20.0 & -- & -- \\ \hline
\hline
\end{tabular}
\caption{\label{tab:courses:teaching2} This table is a summary of the courses I have taught since Fall 2017.  The introductory courses carry the course numbers 135A, 135B, 150, and 180.  The advanced course, PHYS306, is cross-listed as a computer science course (COSC330).}
\end{table}

My first advanced course learning focus is \textbf{mental discipline}, and I attempted to reach that goal in several ways.  First, in Computer Logic and Digital Circuit Design, homeworks were difficult, and assigned in two-week increments, with both mathematical repetition (to facilitate learning to speak with binary and boolean algebra) and open-ended design questions \footnote{See supplemental material for examples of assignments}.  Second, I tried to achieve an efficient content delivery, demonstrating to the students that I was both invested in them and the material.  Teaching binary to newcomers felt like teaching a new programming or spoken language.  I chose to combine a traditional lecture component with electronic slides that meshed with my work on the whiteboard, just like I've observed professors of Spanish at Whittier College.  I had students solve problems individually and in pairs, which helped struggling students during the introduction of Boolean algebra.  Just like a language course requires a student to \textit{verbally communicate} with the professor to improve grammar and comprehension, I required my students to verbally communicate their mathematical ideas to each other.  Finally, I tried to motivate students to dig deeper in understanding through design problems in the homework and group projects.  The students' project designs had to acheive an agreed-upon task, and the design had to be modelled with software, built, tested, and presented to the group.  In addition to mental discipline, the group projects were designed as additional oral and written communication practice, in the form of the presentation.

My second advanced course learning focus is \textbf{strength in all phases of science}.  By design, this course incorporated multiple sub-topics and four phases (as listed above): \textit{mathematics, computer programming, hardware design and function, and computer modelling.}  The first month of the course required me to focus on binary math and boolean algebra, finishing with the topic of Karnaugh maps \footnote{Karnaugh maps are a way of speeding through boolean algebraic derivations efficiently.} Unfortunately, the digital components I had ordered as part of the hardware curriculum did not arrive until four weeks into the semester because of a shipment error.  Thankfully this can only happen once, because most of the parts are reusable.  Thus we focused on simulating the circuits implied by our algebraic derivations with a software package called LogicWorks.  LogicWorks gave the students the benefit of seeing how transistor-based circuits would behave with respect to time.  Once the hardware arrive, we built everything from super-heterodyne AM transistor radios, to circuits that could add two 8-bit binary numbers.  The students enjoyed the tinkering aspect of the course in the lab, but I would have wanted the lab and lecture activities to be more integrated.

My third advanced learning focus is \textbf{communication}.  The final projects were good practice, and first I had groups of two and three submit project proposals to me for approval.  The project proposal rubric is graded on \textit{attention to detail}, and the students responded with diagrams, sketches, and text explaining their design.  Once everyone was on the same page, they had to go and simulate the design, checking it for flaws, and show me progress.  Often times the students would come to my office and ask for help ``de-bugging'' their designs, which is jargon for trouble-shooting.  I recall spending a few hours with each group over the semester getting them to explain design logic, in an attempt to locate the flaw in an algorithm, or to refine a design.  In these moments I admired the growth in the students' mathematical communication.  The final presentations were good, and would have been excellent had I provided more specific guiderails for the presentation content.  What we experienced was two main presentation components: the demonstration and the explanation with data.  In the future, I will formalize the requirement of both, and 

\end{document}

