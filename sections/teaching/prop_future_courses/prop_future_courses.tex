\documentclass[../../main.tex]{subfiles}
 
\begin{document}

\label{sec:future}

Here I propose future courses that will serve the broader liberal arts goals of Whittier College.  I have already submitted two proposals for two CON2-style courses which linked aspects of physics and my sub-field of \textit{astroparticle} physics with broader cultural and scientific issues.  The first was entitled \textit{Physics of the Five Senses} (see below).  The second was a entitled \textit{Safe Return Doubtful: History and Current Status of Modern Science in Antarctica}, which incorporated elements of my sub-field of research with elements of history of science and exploration, environmental science, and climate change.  The latter course would be a longer-term project, whereas the former would require a simple remix of content I already teach.  I have already offered 9.0 credits of COM1 in my first year, because both \textit{calculus-based} introductory courses, PHYS150 and PHYS180, count as COM1.  While this is a start, I plan to do more to help further the liberal arts goals of the College, and the courses below are one set of ideas to meet those goals.  \\ \hspace{0.1cm}

\textbf{Digital Signal Processing (COSC390)} - This special topics in computer science and mathematics is meant to serve 3-2 engineering students, math/computer science majors, and physics/computer science majors, while remaining open to all students interested in interacting with digital data.  The applications will include audio and image processing, and ``analytics'' or Big Data analysis,  This course is a brand new course, to be taught for the first time this coming January term, 2019. There are several advantages to this course that add value for Whittier science and engineering students.  First, this course is a natural continuation of my COSC330/PHYS306, and students taking both should be able to grasp the entire digital-data ecosystem from transistors to data processing.  Second, the course will focus on managing large data sets (Big Data), which prepares students for ``analytics'' applications.  These skills are relevant to a broad range of science courses.  Finally, the coding language (octave)\footnote{See \url{https://www.gnu.org/software/octave}.} and textbook \cite{dsp} for this course will be open source, and therefore free. \\ \hspace{0.1cm}

\textbf{The History of Science in Latin America} - This is a history of science course that I believe would be widely subscribed for several reasons.  Given that a) the ethnic composition of Southern California is changing, and b) Latinos are historically under-represented in physics, it might be helpful for all of our students to become familiar with the scientific achievements of pre-Columbian peoples.  I do see this as part of the broader effor by the college to be more diligent in the areas of equity and inclusion, but the main focus would be on the history of science.  \footnote{Math professors have already suggested the decoding of the Mayan language, and the accuracy of the Mayan calendar as example topics.}  I would center the course on two ideas. First is the idea that those people who make scientific progress have the most accurate data, regardless of where they are on the globe.  For example, civilizations in Latin America saw different celestial objects than those in Europe, and likely had more knowledge of them before European astronomers explored the Southern Hemisphere.  Another example are pre-Columbian Latin American calendars.  These were based on astronomical data, and worked within certain limitations, just as European calendars.  \\ \hspace{0.1cm}

The second idea would be that language matters in science.  I am curious what words pre-Columbian peoples used to desribe certain physics effects, and how they later translated them into Spanish.  As a start, I decided to take introductory Spanish courses with Prof. Doreen O'Conner-G\'{o}mez, who was kind enough to let me audit one of her courses.  I would like to eventually cover original documents from Spanish colonials who had the first glimpse of the scientific knowledge of the indigenous people.  My hypothesis is that indigenous peoples had significant knowledge of physics and astronomy, but that it didn't translate simply into the contemporary European framework.  By understanding both the original Spanish writings, and physical and astronomical effects, I would attempt to show that there was some common understanding of the natural world. This course would probably become a CON2-type course, but I would be open to pairing it as a CON1 if another professor is willing. \\ \hspace{0.1cm}

\textbf{Physics of the Five Senses} - This is a course I proposed as a liberal education breadth course, which would expose non-STEM majors to the physics and kinesiology of the human senses.  Many kinesiology majors take my introductory physics courses to fulfill graduation requirements, and often express an interest in physiological measurements\footnote{See supplemental material for a final project example on muscle activation}.  Why not base an entire course around making physiological measurements, and open it to non-STEM majors who could take it alongside STEM majors?  The course would be fun, and activity-based, centering on KNS and physics labs meant to establish that our five senses are not that different from other sensors based on electrical signals, optics, and thermodynamics.

\end{document}

