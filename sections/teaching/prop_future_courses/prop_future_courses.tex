\documentclass[../../main.tex]{subfiles}
 
\begin{document}

\label{sec:future}

\textbf{Digital Signal Processing (COSC390)} - This special topics in computer science and mathematics is meant to serve primarily 3-2 engineering students, math/computer science majors, and physics/computer science majors.  However it will be open to all students interested in learning more about handling digital data scientifically.  This course is a brand new course I'm developing, to be taught for the first time this coming January term, 2019. \\ \hspace{0.1cm}

I will present a broad overview of the subject of digital signal processing (DSP).  We will begin with a review of relevant statistics, complex numbers, and types of noise in digital systems. Next the concept of a linear DSP system and the corresponding mathematical techniques will be introduced. From there, a broad overview of the topic of digital filters and data processing will be given, proceeding to DSP applications to scientific data analysis. Among the application topics are audio systems and data compression, electrical circuits, digital imaging, and applied neural networks. Time permitting, advanced topics in DSP with complex numbers will be covered, including the Fourier and Laplace transforms with their digital counterparts, the FFT and z-transforms. \\ \hspace{0.1cm}

There are several advantages to this course that add value for Whittier science and engineering students.  This course is a natural continuation of my COSC330/PHYS306 course, Computer Logic and Digital Circuit Design.  Students who have taken both should be able to grasp going all the way from transistor logic to processed digital data.  The course also will focus on cleaning and managing data, and performing data analysis on time-series.  These skills are relevant to a broad range of science courses, in addition to computer science and physics.  Finally, the coding language (octave)\footnote{See \url{https://www.gnu.org/software/octave}.} and textbook \cite{dsp} for this course will be open source, and therefore free for students. \\ \hspace{0.1cm}

\textbf{The History of Science in Latin America} - This is a history of science course I have discussed with professors in my department, and in the Mathematics Department, that I believe would be widely subscribed by the students for several reasons.  Given that a) the ethnic composition of Southern California is changing, and b) Latinos are historically under-represented in physics, it might be helpful for all of our students to become familiar with the scientific achievements of pre-Columbian peoples.  I do see this as part of the broader effor by the college to be more diligent in the areas of equity and inclusion, but the main focus would be on the history of science.  One mathematics professor suggested reviewing the decoding of the Mayan language and scripts.  Another suggested covering the astronomical accuracies of the Mayan calendar.  I would center the course on two ideas. First is the idea that those people who make good scientific progress are the ones who have the most accurate data, regardless of where they are on the globe.  For example, civilizations in Latin America, saw different celestial objects in the sky than those in Europe, and likely had more knowledge of them before European astronomers explored the Southern Hemisphere.  Another example are pre-Columbian Latin American calendars.  These were precise and functional, based on astronomical data, and worked within certain limitations as European medieval calendars.  \\ \hspace{0.1cm}

The second idea would be that language matters in science - often terminology in physics is confined to the English language, however I am curious what words native peoples used to desribe certain physics effects, and how they later translated them into Spanish.  As a start, I decided to take introductory spanish courses with Prof. Doreen O'Conner-G\'{o}mez, who was kind enough to let me audit one of her courses.  I would like to eventually cover original writings and letters from Spanish colonials who had the first glimpse of the scientific knowledge of native Latin Americans.  My hypothesis is that pre-Columbian peoples had significant knowledge of physics and astronomy, but that it didn't necessarily appear so to Europeans who did not speak the same language, or quantify scientific effects in the same way.  By understanding both the original (or perhaps translated) colonial writings, and physical and astronomical effects which are timeless, I would attempt to show that there was some common understanding of the natural world. \\ \hspace{0.1cm}

\textbf{Physics of the Five Senses} - This is a course I proposed as a liberal education breadth course, which would expose non-STEM majors to the physics and kinesiology of the human senses.  Many kinesiology majors take my introductory physics courses to fulfill graduation requirements, and often express an interest in physiological measurements\footnote{See supplemental material for a final project example on muscle activation}.  Why not base an entire course around making physiological measurements, and open it to non-STEM majors who could take it alongside STEM majors?  The course would be fun, and activity-based, centering on KNS and physics labs meant to establish that our five senses are not that different from other sensors based on electrical signals, optics, and thermodynamics.

\end{document}

