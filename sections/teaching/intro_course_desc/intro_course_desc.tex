\documentclass[../../../main.tex]{subfiles}
 
\begin{document}

\begin{table}
\centering
\begin{tabular}{| c | c | c | c | c |}
\hline \hline
Semester & Course & Credits & Students & Curriculum feature \\ \hline
Fall 2017 & PHYS135A-01 & 4.0 & 24 & None \\ \hline
Fall 2017 & PHYS150-01 & 4.0 & 17 & COM1 \\ \hline
Spring 2018 & PHYS135B-01 & 4.0 & 18 & None \\ \hline
Spring 2018 & PHYS180-02 & 5.0 & 19 & COM1 \\ \hline
Spring 2018 & COSC330/PHYS306 & 3.0 & 6 & Advanced course \\ \hline
-- & Total & 20.0 & -- & -- \\ \hline
\hline
\end{tabular}
\caption{\label{tab:courses:teaching} This table is a summary of the courses I have taught since Fall 2017.  The introductory courses carry the course numbers 135A, 135B, 150, and 180.  The advanced course, PHYS306, is cross-listed as a computer science course (COSC330).}
\end{table}

\textbf{\textit{Algebra-based physics (135A/B)}}. Algebra-based physics, PHYS135 A/B, is a two-semester integrated lecture/laboratory sequence that covers algebra-based kinematics, mechanics, and electromagnetism \footnote{See supplemental material for example syllabi.}.  I have taught one section of PHYS135A and one section of PHYS135B, for a total of 42 students.  I employ research-based physics teaching methods, and use the OpenStax open-source textbooks, \textbf{satisfying departmental goals 1, 4, and 6}.  These methods are \textit{Peer Instruction (PI)}, \textit{Just in Time Teaching (JITT)}, and \textit{Physics Education Technology (PhET)}.  In addition to departmental guidance, I attended the American Association of Physics Teachers (AAPT) Workshop for New Professors to learn how to implement these practices \footnote{See supplemental material for details.}.  I attempt to reach the first of the three learning focuses I identify for non-majors, \textbf{basic curiosity}, by actively switching between laboratory and lecture-based activities.  Additionally, I ... ...\\ \hspace{0.1cm}

\textbf{\textit{Calculus-based physics (150/180)}}. Calculus-based physics, PHYS150/PHYS180, is a two-semester sequence that covers calculus-based kinematics, mechanics, thermodynamics, and electromagnetism \footnote{See supplemental material for example syllabi.}.  I have taught one section of PHYS150 and one section of PHYS180, for a total of 36 students.  As in the algebra-based classes, I implement \textit{Peer Instruction (PI)}, \textit{Just in Time Teaching (JITT)}, and \textit{Physics Education Technology (PhET)}, and use OpenStax textbooks.  The key difference between calculus and algebra-based physics methods is the increased use of PhET simulations to visualize calculus concepts.  Because PHYS150 and PHYS180 require tools from single and multi-variable calculus, students taking those courses concurrently require PhET simulations to help visualize mathematical concepts.  Examples include operations with scalar and vector fields in electromagnetism, and single-variable integrals and derivatives in kinematics. \\ \hspace{0.1cm}

\underline{PI Modules} - Implementation of an active learning strategy involving group problem solving.
\begin{itemize}
\item PI-based modules are centered around posing conceptual, multiple-choice questions to the class about a physical system.  
\item Students respond individually with an electronic device, and the distribution of answers for choices A-D is shown on the class screen.
\item One of two actions is taken next:
\begin{enumerate}
\item If the fraction of correct answers to the conceptual question is larger than 0.7, the class moves forward in the material.
\item If the fraction of correct answers to the conceptual question is less than 0.7, the professor initiates table discussion.
\end{enumerate}
\item Table discussions take place between 2-4 students at the same table.  The professor tells the students to \textit{attempt to convince each other they are right, and that just because they gave the same answer does not indicate correctness\footnote{The effect of adding this specific phrase has been studied and shown to benefit the utility of table discussions.}.}
\item A second poll of the class is taken, to measure the increased fraction of correct answers, or \textit{gain.} Often I include E as an answer in addition to A-D, to be selected (anonymously) by students who are totally confused.  If more than one person selects E after the second round, I pause to go over the material a second time.
\end{itemize}

\underline{JITT Modules} - Modification of lecture time based on student reading the day before class.
\begin{itemize}
\item JITT activities grew out of reading quizzes in a traditionally structured course.  Through Moodle, I message students the day prior to class 3-4 questions based on the reading I asked them to complete prior to class time.
\item JITT questions are conceptual, and if a large portion of students are answering correctly, the material is covered more lightly.  Questions that trigger many incorrect responses becomes the focus of class time.
\item Because I keep all of my PI-module questions in a database, I attempt to locate questions tailored to the misconceptions.
\item One interesting facet of the JITT-style is that students' anonymous responses are included in the lecture itself, and the class gets a chance to analyze correct and incorrect responses.
\end{itemize}

\underline{PhET Modules} - Simulation activities integrated into the textbook and laboratory/PI modules.
\begin{itemize}
\item The OpenStax textbooks for PHYS135 and PHYS150/PHYS180 have built-in HTML links to JAVA-based simulations called PhET simulations\footnote{see \url{https://phet.colorado.edu}}.
\item I incorporate PhET simulations into laboratory activities, in which simulated results of a system are compared to measurements of identical systems in the lab.
\item For physics systems we cannot build in the lab, we use the PhET activities as measurements to compare to calculations.
\item PhET simulations often augment special curricular activities pertaining to other majors, like the human body.  For example, in PHYS135B we used a PhET simulation to understand the behavior of human nerve signals.
\end{itemize}

\end{document}

